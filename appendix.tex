\section{Advanced Setup: HTTP Image Fetching}
\label{sec:http}

\label{sec:testbed_http}


\subsection{HTTP}


\begin{lstlisting}[]
$ export HTTPBOOTIMAGE="${WORKDIR}/httpboot"
$ export HTTPROOTIMAGE="${WORKDIR}/httproot"
\end{lstlisting}
\FloatBarrier
\vspace{-5mm}

Extract root partition from image
\begin{lstlisting}[]
# export DEV=$(sudo losetup --show -f -P ${IMAGE}.img); echo $DEV
# dd if=${DEV}p1 of=${HTTPBOOTIMAGE}.img bs=4M
# dd if=${DEV}p2 of=${HTTPROOTIMAGE}.img bs=4M
\end{lstlisting}
\FloatBarrier
\vspace{-5mm}


Mount http root and home directories (ADD LOOP OPTION HERE WHEN MOUNTING)
\begin{lstlisting}[]
$ export HTTPROOTDIR="${WORKDIR}/httproot"
$ mkdir -p ${HTTPROOTDIR}
# mount ${HTTPROOTIMAGE}.img ${HTTPROOTDIR}
# mount ${HTTPBOOTIMAGE}.img ${HTTPROOTDIR}/boot
\end{lstlisting}
\FloatBarrier
\vspace{-5mm}


Create the following file (THE READ-ONLY OPTION IS STILL NOT CONFIGURABLE IN THIS SCRIPT)
\Suppressnumber\begin{lstlisting}[]
<@\textcolor{gray}{\$\{HTTPROOTDIR\}/etc/initramfs-tools/scripts/http}@>
<@\textcolor{gray}{
---------------------------------------------------------------}
\Reactivatenumber @>
# Fetch, cache, and mount root image using HTTP

#. /scripts/functions # Already loaded in the init script

# Create mount point directory
[ -d /imagedir ] || mkdir /imagedir

# Parse additional command line options
for x in $(cat /proc/cmdline); do
    case $x in
    image=*)
        IMAGE=${x#image=}
        # Extract device and path
        # E.g. /dev/mmcblkp2:/root.img --> /dev/mmcblkp2 and /imagedir/root.img
        IMAGEDEV=${IMAGE%:*}
        IMAGEPATH="${IMAGE#*:}"
        IMAGEPATH="/imagedir/${IMAGEPATH#/}"
        ;;
    url=*)
        URL=${x#url=}
        URL=${URL%/}  # If url has trailing slash, then remove it
        ;;
    esac
done

# Check if variables are declared
[ -z "${IMAGE}" ] && echo "Error: Please add 'image' in cmdline.txt" && exit 1
[ -z "${URL}" ] && echo "Error: Please add 'url' in cmdline.txt" && exit 1

# Fetch and cache root image
http_fetch_image()
{
    configure_networking

    # Fetch list of root image names
    if ! wget -O /tmp/image.csv "${URL}/image.csv"; then
        echo "WARNING failed to fetch image list"
        return
    fi

    # Extract name of image to download based on eth0 MAC
    mac=$(cat /sys/class/net/eth0/address)
    img=$(grep $mac /tmp/image.csv | cut -f2 -d' ')

    # If the image is not cached, then get it
    if [ ! -f ${img} ] && wget --spider -q "${URL}/${img}"; then
        # Delete old image
        rm -f $(readlink -f ${IMAGEPATH})

        # Fetch new image
        if wget -O "${IMAGEPATH%/*}/${img}" "${URL}/${img}"; then
            # If success, create symbolic link to new image
            ln -f -s "${IMAGEPATH%/*}/${img}" ${IMAGEPATH}
        fi
    fi

    # Cleanup
    #rm /tmp/image.csv
}

http_mount_root()
{
    modprobe ext4

    # Mount imagedir
    #mount -n -t ext4 ${IMAGEDEV} /${${IMAGEPATH#/}%%/*}
    mount -n -t ext4 ${IMAGEDEV} /imagedir

    # Fetch and cache root image
    http_fetch_image

    # Mount root as readonly
    losetup /dev/loop0 $(readlink -f ${IMAGEPATH})
    mount -r -n -t ext4 -o nodiratime,noatime /dev/loop0 /root

    # Bind /root/home to /home
    #mount -n -o rbind /home /root/home
}

mountroot()
{
    http_mount_root
}
\end{lstlisting}
\FloatBarrier
\vspace{-5mm}

Make the script executable
\begin{lstlisting}[]
# chmod +x ${HTTPROOTDIR}/etc/initramfs-tools/scripts/http
\end{lstlisting}
\FloatBarrier
\vspace{-5mm}

Add the below to \$\{HTTPROOTDIR\}/boot/cmdline.txt (CHANGE URL HERE) (remove init=/usr/lib/raspi-config/init\_resize.sh earlier in guide)
\begin{lstlisting}[]
boot=http ro ip=dhcp image=/dev/mmcblk0p2:/root.img root=/dev/loop0 url=http://192.168.87.106/raspi/
\end{lstlisting}
\FloatBarrier
\vspace{-5mm}
% ip=dhcp root=/dev/http http-options=hard,intr,ro


Edit fstab to look like this (exclude the last line if persistent
home storage directory is not desired):
\Suppressnumber\begin{lstlisting}[]
<@\textcolor{gray}{\$\{HTTPROOTDIR\}/etc/fstab}@>
<@\textcolor{gray}{
---------------------------------------------------------------}
\Reactivatenumber @>
proc            /proc           proc    defaults          0       0
/dev/mmcblk0p1  /boot           vfat    defaults          0       2
/dev/loop0      /               ext4    defaults,noatime  0       1
/dev/mmcblk0p2  /home           ext4    defaults,noatime  0       0
\end{lstlisting}
\FloatBarrier
\vspace{-5mm}


Chroot into the image
% CHROOT to OS image
\begin{lstlisting}[]
$ cd $HTTPROOTDIR
# mount -t proc proc proc/
# mount --bind /sys sys/
# mount --bind /dev dev/
# mount --bind /dev/pts dev/pts
# proot -q qemu-arm-static -S ${HTTPROOTDIR}
# export PS1="(chroot) $PS1"
\end{lstlisting}
\FloatBarrier
\vspace{-5mm}


Create new initramfs (remember that your kernel might be different)
\begin{lstlisting}[]
(chroot) # mkinitramfs -o /boot/init.gz -k 4.4.11+
\end{lstlisting}
\FloatBarrier
\vspace{-5mm}

Exit chroot
\begin{lstlisting}[]
(chroot) # exit
\end{lstlisting}
\FloatBarrier
\vspace{-5mm}

Unmount:
\begin{lstlisting}[]
# umount ${HTTPROOTIMAGE}/{dev/pts,proc,sys,dev}  
\end{lstlisting}
\FloatBarrier
\vspace{-5mm}


\subsubsection{Prepare sd-card}

IS SETTING UP THE SD-CARD REALLY THE SAME PROCEDURE AS FOR A LOCAL FILESYSTEM?

% Define variable
% export IMAGE="2016-05-27-raspbian-jessie-lite"
% export WORKDIR="${HOME}/Raspbian"
\begin{lstlisting}[]
$ export HTTPDEV="/dev/mmcblk0"
\end{lstlisting}
\FloatBarrier
\vspace{-5mm}

MAYBE ALSO SET THE BOOT FLAG FOR FIRST PARTITION. ALSO UPDATE SCRIPT IF THIS
WORKS. THERE MAY HAVE TO BE ECHOED A 1 AFTER echo a FOR SETTING BOOT FLAG
\begin{lstlisting}[]
<@\Suppressnumber @># (echo o;                                      # Create DOS partition table
    echo n; echo p; echo 1; echo ; echo +64M;   # Create boot partition
    echo t; echo b;                             # Set boot partition type to FAT
    echo a;                                     # Set boot flag
    echo n; echo p; echo 2; echo ; echo ;       # Create home partition
    echo w                                      # Write table to disk
    ) | sudo fdisk ${HTTPDEV} <@\Reactivatenumber @>
\end{lstlisting}
\FloatBarrier
\vspace{-5mm}

Print partition table (SHOULD THE BOOT FLAG HAVE BEEN SET TO PARTITION P1?)
\begin{lstlisting}[]
# fdisk -u sectors -l ${HTTPDEV}
Disk /dev/mmcblk0: 3.7 GiB, 3965190144 bytes, 7744512 sectors
Units: sectors of 1 * 512 = 512 bytes
Sector size (logical/physical): 512 bytes / 512 bytes
I/O size (minimum/optimal): 512 bytes / 512 bytes
Disklabel type: dos
Disk identifier: 0x00e61b04

Device         Boot  Start     End Sectors  Size Id Type
/dev/mmcblk0p1        2048  133119  131072   64M  b W95 FAT32
/dev/mmcblk0p2      133120 7744511 7611392  3.6G 83 Linux
\end{lstlisting}
\FloatBarrier
\vspace{-5mm}

Format partitions
\begin{lstlisting}[]
# mkfs.vfat -n BOOT ${HTTPDEV}p1
# mkfs.ext4 -L HOME ${HTTPDEV}p2
\end{lstlisting}
\FloatBarrier
\vspace{-5mm}






Mount the boot partition from the memory card. Copy the content from the
boot image and unmount it again:
\begin{lstlisting}[]
# mount ${HTTPDEV}p1 ${HTTPROOTDIR}/mnt
# cp -r ${HTTPROOTDIR}/boot/* ${HTTPROOTDIR}/mnt/ && sync
# umount ${HTTPROOTDIR}/mnt
\end{lstlisting}
\FloatBarrier
\vspace{-5mm}

Mount the home partition from the memory card. Copy the content from the
root image and unmount it again:
\begin{lstlisting}[]
# mount ${HTTPDEV}p2 ${HTTPROOTDIR}/mnt
# cp -r ${HTTPROOTDIR}/home/* ${HTTPROOTDIR}/mnt/ && sync
# umount ${HTTPROOTDIR}/mnt
\end{lstlisting}
\FloatBarrier
\vspace{-5mm}


\begin{lstlisting}[]
# umount --recursive ${HTTPROOTDIR}
\end{lstlisting}
\FloatBarrier
\vspace{-5mm}


\subsubsection{(Optional) Compressing root image}

PROBABLY A BETTER IDEA TO USE BTRFS TRANSPARENT COMPRESSION

\begin{lstlisting}[]
# apt-get install squashfs-tools
# mount -o loop ${HTTPROOTIMAGE}.img ${HTTPROOTDIR} 
# mksquashfs ${HTTPROOTDIR} compressed_${HTTPROOTIMAGE}.img       
\end{lstlisting}
\FloatBarrier
\vspace{-5mm}

\subsubsection{Try it out}

Upload the root image (\$\{HTTPROOTIMAGE\}.img) to the HTTP server used in the
configurations above. Then, unplug the memory card from the computer.
Insert it into a \ac{Raspi} and
turn it on. It is recommended that the \ac{Raspi} is turned on with a monitor
connected the first time the newly created Linux installation is tested to
assure that Raspbian starts up correctly.


PRESENT FILES ON HTTP. E.G. HTTPROOT.IMG AND IMAGES.CSV





\section{Alternative Commands for Outdated Packages}
This section describes alternative commands in case the testbed administrator
is not able to update old packages in its Linux distribution for performing
the commands, particularly the ones regarding the \texttt{util-linux} package.

\subsection{Losetup for Loopback Devices}
\label{sec:alternative_losetup}

In case the \texttt{losetup -P} is not available given that a version
higher or equal than 2.21 for \texttt{util-linux} is neither not available,
a possible option is to manually set all the loopback devices used during the
whole procedure with \texttt{mknod}. To do this, the alternative commands are:

\begin{lstlisting}[]
$ export DEV="/dev/loop0"
# mknod ${DEV}p1 b 7 1
# mknod ${DEV}p2 b 7 2
# losetup -o $((8192*512)) --sizelimit $((137215*512)) ${DEV}p1 ${IMAGE}.img
# losetup -o $((137216*512)) --sizelimit $((4806655*512)) ${DEV}p2 ${IMAGE}.img
\end{lstlisting}
\FloatBarrier
\vspace{-5mm}

However, there will be a few differences by using the above code as an
alternative to the \texttt{losetup -P} case. First, the output from
\texttt{lsblk} will look different:

\begin{lstlisting}[]
# lsblsk
  NAME   MAJ:MIN RM   SIZE RO TYPE MOUNTPOINT
  sda      8:0    0 238.5G  0 disk
    sda1   8:1    0   512M  0 part /boot
    sda2   8:2    0   168G  0 part /
    sda3   8:3    0     2G  0 part [SWAP]
    sda4   8:4    0    68G  0 part
  loop1    7:1    0    67M  0 loop
  loop2    7:2    0   2.2G  0 loop
\end{lstlisting}
\FloatBarrier
\vspace{-5mm}

Second, detaching the loop devices will also be different:

\begin{lstlisting}[]
# losetup -d ${DEV}p{1,2}
\end{lstlisting}
\FloatBarrier
\vspace{-5mm}

Working with these loopback devices will be transparent for the remainding
procedure.

Also, when creating the persistent home directory partition in
Section~\ref{sec:persistent_directories}, for recreating the loopback devices
the commands are:

\begin{lstlisting}[]
$ export DEV="/dev/loop0"
# mknod ${DEV}p2 b 7 2
# mknod ${DEV}p3 b 7 3
# losetup -o $((137216*512)) --sizelimit $(( (4806655-137216+1)*512)) ${DEV}p2 ${IMAGE}.img
# losetup -o $((4806656*512)) --sizelimit $(( (6903807-4806656+1)*512)) ${DEV}p3 ${IMAGE}.img
\end{lstlisting}
\FloatBarrier
\vspace{-5mm}

Finally, for detaching in this case the command is:
\begin{lstlisting}[]
# losetup -d ${DEV}p{2,3}
\end{lstlisting}
\FloatBarrier
\vspace{-5mm}


\subsection{Umount after Image Chroot}
\label{sec:umount}

In case the \texttt{umount --recursive} is not available given that a version
higher or equal than 2.22 for \texttt{util-linux} is neither not available,
a possible option is to manually umount all the partitions used during the
\texttt{chroot} environment. The directories must be unmounted in the reverse
order as when they were mounted. These commands are:

\begin{lstlisting}[]
# umount ${ROOTDIR}/dev/pts
# umount ${ROOTDIR}/dev
# umount ${ROOTDIR}/sys
# umount ${ROOTDIR}/proc
\end{lstlisting}
\FloatBarrier
\vspace{-5mm}