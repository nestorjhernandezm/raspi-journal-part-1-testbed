\section{Alternative for losetup}
\label{sec:alternative_losetup}

\begin{lstlisting}[]
# export DEV="${losetup -f}"
/dev/loop0
\end{lstlisting}
\FloatBarrier
\vspace{-5mm}

\begin{lstlisting}[]
# mknod -m 660 ${DEV}p1 b 7 1
# mknod -m 660 ${DEV}p2 b 7 2
\end{lstlisting}
\FloatBarrier
\vspace{-5mm}

\begin{lstlisting}[]
# fdisk -lu ${IMAGE}.img
WHAT WILL BE THE OUTPUT?
\end{lstlisting}
\FloatBarrier
\vspace{-5mm}

I THINK SIZELIMIT IS NOT REQUIRED
\begin{lstlisting}[]
# losetup -o $((8192*512)) --sizelimit $((137215*512)) ${DEV}p1 ${IMAGE}.img
# losetup -o $((137216*512)) --sizelimit $((4806655*512)) ${DEV}p2 ${IMAGE}.img
\end{lstlisting}
\FloatBarrier
\vspace{-5mm}

The output from \texttt{lsblk} will look a bit different

\section{Alternative for Recursive Unmount after Image Chroot}
\label{sec:umount}

In case the \texttt{umount --recursive} is not available given that a version
higher or equal than 2.22 for \texttt{util-linux} is neither not available,
a possible option is to manually umount all the partitions used during the
\texttt{chroot} environment.

\begin{lstlisting}[]
# losetup -o $((8192*512)) --sizelimit $((137215*512)) ${DEV}p1 ${IMAGE}.img
# losetup -o $((137216*512)) --sizelimit $((4806655*512)) ${DEV}p2 ${IMAGE}.img
\end{lstlisting}
\FloatBarrier
\vspace{-5mm}

\section{http}
\label{sec:http}

The HTTP description

