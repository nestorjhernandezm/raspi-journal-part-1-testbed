\section{Advanced Setup: HTTP Image Fetching}
\label{sec:http}

The HTTP description

\section{Alternative Commands for Outdated Packages}
This section describes alternative commands in case the testbed administrator
is not able to update old packages in its Linux distribution for performing
the commands, particularly the ones regarding the \texttt{util-linux} package.

\subsection{Losetup for Loopback Devices}
\label{sec:alternative_losetup}

In case the \texttt{losetup -P} is not available given that a version
higher or equal than 2.21 for \texttt{util-linux} is neither not available,
a possible option is to manually set all the loopback devices used during the
whole procedure with \texttt{mknod}. To do this, the alternative commands are:

\begin{lstlisting}[]
$ export DEV="/dev/loop0"
# mknod ${DEV}p1 b 7 1
# mknod ${DEV}p2 b 7 2
# losetup -o $((8192*512)) --sizelimit $((137215*512)) ${DEV}p1 ${IMAGE}.img
# losetup -o $((137216*512)) --sizelimit $((4806655*512)) ${DEV}p2 ${IMAGE}.img
\end{lstlisting}
\FloatBarrier
\vspace{-5mm}

However, there will be a few differences by using the above code as an
alternative to the \texttt{losetup -P} case. First, the output from
\texttt{lsblk} will look different:

\begin{lstlisting}[]
# lsblsk
  NAME   MAJ:MIN RM   SIZE RO TYPE MOUNTPOINT
  sda      8:0    0 238.5G  0 disk
    sda1   8:1    0   512M  0 part /boot
    sda2   8:2    0   168G  0 part /
    sda3   8:3    0     2G  0 part [SWAP]
    sda4   8:4    0    68G  0 part
  loop1    7:1    0    67M  0 loop
  loop2    7:2    0   2.2G  0 loop
\end{lstlisting}
\FloatBarrier
\vspace{-5mm}

Second, detaching the loop devices will also be different:

\begin{lstlisting}[]
# losetup -d ${DEV}p{1,2}
\end{lstlisting}
\FloatBarrier
\vspace{-5mm}

Working with these loopback devices will be transparent for the remainding
procedure.

Also, when creating the persistent home directory partition in
Section~\ref{sec:persistent_directories}, for recreating the loopback devices
the commands are:

\begin{lstlisting}[]
$ export DEV="/dev/loop0"
# mknod ${DEV}p2 b 7 2
# mknod ${DEV}p3 b 7 3
# losetup -o $((137216*512)) --sizelimit $(( (4806655-137216+1)*512)) ${DEV}p2 ${IMAGE}.img
# losetup -o $((4806656*512)) --sizelimit $(( (6903807-4806656+1)*512)) ${DEV}p3 ${IMAGE}.img
\end{lstlisting}
\FloatBarrier
\vspace{-5mm}

Finally, for detaching in this case the command is:
\begin{lstlisting}[]
# losetup -d ${DEV}p{2,3}
\end{lstlisting}
\FloatBarrier
\vspace{-5mm}


\subsection{Umount after Image Chroot}
\label{sec:umount}

In case the \texttt{umount --recursive} is not available given that a version
higher or equal than 2.22 for \texttt{util-linux} is neither not available,
a possible option is to manually umount all the partitions used during the
\texttt{chroot} environment. The directories must be unmounted in the reverse
order as when they were mounted. These commands are:

\begin{lstlisting}[]
# umount ${ROOTDIR}/dev/pts
# umount ${ROOTDIR}/dev
# umount ${ROOTDIR}/sys
# umount ${ROOTDIR}/proc
\end{lstlisting}
\FloatBarrier
\vspace{-5mm}