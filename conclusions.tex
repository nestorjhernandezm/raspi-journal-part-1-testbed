\label{sec:conclusions}
Given the need for low-cost, easy to configure, testbeds that enable
reproducible results, we provide a in-depth description of Aalborg
University's \ac{Raspi} testbed and how to guarantee replicable
results and managing scaling of the system.  We also provide a
performance evaluation for network coding schemes, focusing on
processing speed and energy behaviour for two \ac{Raspi} models,
including algorithms that exploit \ac{SIMD} instructions and multicore
capabilities of the \ac{Raspi} 2. Our measurements show that
processing speeds of more than 80 Mbps and 800 Mbps are attainable for
the \ac{Raspi} model 1 and 2, respectively, for a wide range of
network coding configurations and maintaining a processing energy
below 1 nJ/bit (or even an order of magnitude lower) in similar
configurations. Future work in the use of \ac{Raspi} devices will
focus on expanding the setup and automation of tasks to run the
testbed, configure specified network topologies (e.g., with specific
connectivity or packet loss ratios), reserve the use of these
sub-networks for running tailored experiments, and opening the use of
the testbed beyond our team at Aalborg University.
