\label{sec:conclusions}
Observing the expectation of the \ac{IoT} and lack for a low-cost, easy to configure testbed that enables reproducible results for mechanisms to address problems in this area, we provide a in-depth description of the new Aalborg University's \ac{Raspi} testbed for network coding evaluation and how to guarantee replicability and scaling managing of this system. The description shows how to set up interconnected \ac{Raspi}s with small SD cards for local storage, a Raspberry distro image, network connectivity and proper system administration privileges. The exposed work will permit researchers to replicate setups and scenarios for evaluating their strategies in a rapid and manageable way. Future work in the use of \ac{Raspi} devices will focus on expanding the setup and automation of tasks to run the testbed, configure specified network topologies (e.g., with specific connectivity or packet loss ratios), reserve the use of these sub-networks for running tailored experiments and opening the use of the testbed beyond our team at Aalborg University.
