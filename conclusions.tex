\label{sec:conclusions}
Observing the expectation of the \ac{IoT} and lack for a low-cost, easy-to-configure testbed in this area for reproducible research, we provide an in-depth description of the new Aalborg University's \ac{Raspi} testbed for network coding evaluation and how to guarantee replicability and scaling managing of this system. The description shows how to set up interconnected \ac{Raspi}s with memory cards for local storage, a Raspbian Lite image, network connectivity and proper system administration privileges. Using the presented procedure permits to setup a Raspbian Lite for the \ac{Raspi}s, but this restrics the testbed administrators to specifically using this \ac{OS} regardless of the \ac{Raspi} model. Other Linux distributions might be setup with Yocto to create an image from the scratch that are tailored for the \ac{Raspi}, even to specific models. However, that would require to compile the software for the \ac{Raspi} from scratch which can be a tedious angd long task. Still, this method could be adequate for an expert user. We hope this work permits researchers to replicate setups and scenarios for evaluating their strategies in a rapid and manageable way. Future work in the use of \ac{Raspi} devices will focus on expanding the setup and automation of tasks to run the testbed, configure specified network topologies (e.g., with specific connectivity or packet loss ratios), reserve the use of these sub-networks for running tailored experiments and opening the use of the testbed beyond our team at Aalborg University. Future work in this area will consider to make the testbed fetch the image through the \ac{HTTP} server. This is expected to simplify the maintenance of the memory cards.
