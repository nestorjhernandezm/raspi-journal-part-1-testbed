\label{sec:cross_compilation}
An important case of a computational expensive task is to compile software
packages and large libraries. Given the computing capabilities of the
\ac{Raspi}, such tasks can be challenging if not prohibitive in terms of
\ac{CPU}, memory or space usage and / or compilation time. Thus, in
this Section we present a procedure of how to cross-compile C++
source code from the testbed administrator \ac{PC} for the \ac{ARM}
architecture of the \ac{Raspi}s. By doing this, we take advantage
of the (typically) much higher computing power of the testbed
administrator \ac{PC} in order to save time and computational resources.
Hence, we give an example of compiling a simple C++ program and
copying the generated binaries with \ac{SSH} to run locally
on a \ac{Raspi}.

Furthermore, given that our testbed purpose is for
network coding applications, we also present how to cross-compile Kodo
\cite{kodo2011pedersen}, a C++11 network coding library to perform
encoding, decoding and recoding operations. In this way, we aim to present
a fully configurable and manageable testbed with the capabilities to
evaluate network coding protocols with several \ac{Raspi}s and locally
store measurements from different evaluations. Therefore, we also show
how \texttt{kodo-cpp}, a set of high-level C++ bindings for Kodo, can be
cross-compiled for applications with the \ac{Raspi}.

\subsection{Toolchain Setup}
To compile in a given architecture that is aimed for a different one,
the testbed administrator needs to install a toolchain in its \ac{PC}.
The toolchain is mandatory due to the different processor architectures
where the source can be compiled from.
Given that compiling a toolchaing can be an ardous task, we get the toolchain
recommended for the \ac{ARM} architecture of the \ac{Raspi}s.
This toolchain is available from \cite{2016rpitoolchain} and
it already contains the binaries for different compilers based on
\texttt{gcc 4.9}. In what follows, we extract the binaries adjusting
them to our coding style and compiling convention. For this, we use the
\texttt{\$TOOLCHAIN} directory as the working directory. However,
the testbed administrator may choose some other working directory of
its preference if desired. First, we create the toolchain directory:

\begin{lstlisting}[]
$ export TOOLCHAINDIR="${HOME}/toolchains"
$ mkdir -p $TOOLCHAINDIR
$ cd $TOOLCHAINDIR
\end{lstlisting}
\FloatBarrier
\vspace{-5mm}

Later, we download the \ac{Raspi} toolchain for 64-bit Linux. However,
instead of working with the binaries from \cite{2016rpitoolchain},
we use a zipped version with the original binaries and renamed symbolic
links to the original files available in
\cite{2016steinwurfrpitoolchain}. Finally, we unzip the downloaded file.
This is made as follows:

\begin{lstlisting}[]
$ wget http://files.steinwurf.com/toolchains/linux/raspberry-gcc-4.9.3/raspberry-gxx493-arm.zip
$ unzip raspberry-gxx493-arm.zip
\end{lstlisting}
\FloatBarrier
\vspace{-5mm}

Instead of calling the ARM cross compiler using its full path, we make
the binaries accessible from the command shell systemwide. A way to do
this is by adding the following commands in the \texttt{\${HOME}/.profile}
as follows:

\begin{lstlisting}[]
$ sed -i '$a export TOOLCHAINDIR=\"$HOME/toolchains\"' ${HOME}/.profile
$ sed -i '$a export TOOLCHAINBINARY=\"raspberry-gxx49-arm-g++\"' ${HOME}/.profile
$ sed -i '$a PATH=\"\$PATH:${TOOLCHAINDIR}/arm-rpi-4.9.3-linux-gnueabihf/bin\"' ${HOME}/.profile
\end{lstlisting}
\FloatBarrier
\vspace{-5mm}

This helps the \ac{OS} to recognize the location of the compiler
command everytime a new shell is opened. The \texttt{.profile} should now
contain the lines we inserted. There might be other
code in the file of other testbed administrators. We recommend to leave
other parts unmodified.
% MAC and Hostname file
\Suppressnumber\begin{lstlisting}[]
<@\textcolor{gray}{\$HOME/.profile}@>
<@\textcolor{gray}{
---------------------------}
\Reactivatenumber @>
...
export TOOLCHAINDIR="$HOME/toolchains"
export TOOLCHAINBINARY="raspberry-gxx49-arm-g++"
PATH="$PATH:${TOOLCHAINDIR}/arm-rpi-4.9.3-linux-gnueabihf/bin"
...
\end{lstlisting}
\FloatBarrier
\vspace{-5mm}

To update the \texttt{PATH} variable and the \texttt{.profile}, we use the
\texttt{source} command for the changes take effect in the administrator
system:
\begin{lstlisting}[]
$ source ${HOME}/.profile
\end{lstlisting}
\FloatBarrier
\vspace{-5mm}

A working ARM cross-compiler in the testbed administrator \ac{PC} should
output the following:

\begin{lstlisting}[]
$ ${TOOLCHAINBINARY} --version
raspberry-gxx49-arm-g++ (crosstool-NG crosstool-ng-1.22.0-88-g8460611) 4.9.3
Copyright (C) 2015 Free Software Foundation, Inc.
This is free software; see the source for copying conditions.  There is NO
warranty; not even for MERCHANTABILITY or FITNESS FOR A PARTICULAR PURPOSE.
\end{lstlisting}
\FloatBarrier
\vspace{-5mm}

\subsection{Cross-compile Example}
The following shows: (i) How to cross compile the classic \texttt{hello\_world}
C++ example for the \ac{Raspi} ARM architecture and (ii) how to copy and
execute the binary in a \ac{Raspi} using \ac{SCP} and \ac{SSH}. First, we
create the file \texttt{hello\_world.cpp}. For simplicity, we create it
in the directory where we stored the \texttt{fabfile.py} file
with the following content using any text editor:

\Suppressnumber\begin{lstlisting}[]
<@\textcolor{gray}{\${CODEDIR}/hello\_world.cpp}@>
<@\textcolor{gray}{
-----------------------------------------------}
\Reactivatenumber @>
#include <iostream>

int main()
{
    std::cout << "Hello World!" << std::endl;
    return 0;
}
\end{lstlisting}
\FloatBarrier
\vspace{-5mm}

We save the previous file and compile it for \ac{Raspi} in the testbed
administrator \ac{PC} by doing:

\begin{lstlisting}[]
$ ${TOOLCHAINBINARY} hello_world.cpp -o hello_world
\end{lstlisting}
\FloatBarrier
\vspace{-5mm}

This should produce a binary \texttt{hello\_world} that should only be
useful in the \ac{Raspi}. We copy it to a \ac{Raspi} to test it. We will
use \ac{SCP} to copy the binary file to one of the \ac{Raspi}s. Still, we
for example might use Fabric if we are interested in deploying a compiled
binary for many \ac{Raspi}s.

\begin{lstlisting}[]
$ scp hello_world pi@<RASP_IP>:~/
\end{lstlisting}
\FloatBarrier
\vspace{-5mm}

% If you do not know the \ac{IP} address of a \ac{Raspi} in your network, you can
% connect a monitor to it and run the following command after you have logged in:
% \begin{lstlisting}[]
% pi@rasp01:~ $ ifconfig
% eth0      Link encap:Ethernet  HWaddr b8:27:eb:72:77:54
%           inet addr:192.168.87.107  Bcast:192.168.87.255  Mask:255.255.255.0
%           inet6 addr: fe80::e0a5:38f3:6f82:bc79/64 Scope:Link
%           UP BROADCAST RUNNING MULTICAST  MTU:1500  Metric:1
%           RX packets:1537 errors:0 dropped:0 overruns:0 frame:0
%           TX packets:445 errors:0 dropped:0 overruns:0 carrier:0
%           collisions:0 txqueuelen:1000
%           RX bytes:259117 (253.0 KiB)  TX bytes:52551 (51.3 KiB)

% \end{lstlisting}
% \FloatBarrier
% \vspace{-5mm}

After the executable has been copied to the \ac{Raspi}. We login through
\ac{SSH} to it:

\begin{lstlisting}[]
$ ssh pi:<RASP_IP>
\end{lstlisting}
\FloatBarrier
\vspace{-5mm}

We can list the directory content after we have logged into the \ac{Raspi} and
verify that the compiled \texttt{hello\_world} binary is there:

\begin{lstlisting}[]
pi@<RASP_IP>:~ $ ls
hello_world  rasp_config
\end{lstlisting}
\FloatBarrier
\vspace{-5mm}

Finally, we simply execute the \texttt{hello\_world} to confirm that
the cross-compiling of \texttt{hello\_world} worked properly:

\begin{lstlisting}[]
pi@<RASP_IP>:~ $ ./hello_world
Hello World!
\end{lstlisting}
\FloatBarrier
\vspace{-5mm}

\subsection{Cross-compile Kodo}

As we originally mentioned, Kodo is a C++11 network coding library that
permits to implement network coding functionalities by allowing any network
protocol designer to use and test the primitive encoding, decoding and
recoding operations of \ac{RLNC}. In this way, a designer only needs to
focus on the design and test of a network coding-based protocol. Kodo is
available through programming bindings for a variety of popular
programming languages. This procedure will present how to configure the
Kodo C++ bindings \texttt{kodo-cpp} to cross-compile applications that can
run in \ac{Raspi}. \texttt{kodo-cpp} provides a simple interface to the
underlying C++11 code that exists in the libraries \texttt{kodo-core} for
the object structure and \texttt{kodo-rlnc} for the \ac{RLNC} codec
implementation. More details about Kodo are provided in the code
documentation~\cite{kodocppdoc}.

To use Kodo for research, it is required to obtain a research free license.
To do this, only a request form needs to be filled
in~\cite{steinwurflicenselink} and wait for grant access. Once
the access for Kodo has been granted, the source code can be pulled from
its Git repositories to be compiled. Thus, assuming that testbed
administrator already has access, we clone the \texttt{kodo-cpp}
repository locally in \texttt{\${CODEDIR}} and change directory into
the repository by doing:

\begin{lstlisting}[]
$ cd ${CODEDIR}
$ git clone git@github.com:steinwurf/kodo-cpp.git
$ cd kodo-cpp
\end{lstlisting}
\FloatBarrier
\vspace{-5mm}

We first configure \texttt{kodo-cpp} to build executables for the
\ac{ARM} architecture using the \ac{Raspi} toolchain and later build them
by running:
\begin{lstlisting}[]
$ python waf configure --cxx_mkspec=cxx_raspberry_gxx49_arm
...
'configure' finished successfully (X.XXXs)
$ python waf build
...
'build' finished successfully (XmXX.XXs)
\end{lstlisting}
\FloatBarrier
\vspace{-5mm}

If the configuration and build steps are successful, the binaries should
have been created. To be able to use them, we need to create a shared library
that we will use in the \ac{Raspi}. To do this, we run the following command:

\begin{lstlisting}[]
$ python waf install --install_shared_libs --install_path="./shared_test"
...
'install' finished successfully (X.XXXs)
\end{lstlisting}
\FloatBarrier
\vspace{-5mm}

Now, we copy the shared library, binary files and related headers to the
\ac{Raspi} home directory as follows:

\begin{lstlisting}[]
$ scp -r shared_test/include shared_test/libkodoc.so pi@<RASP_IP>:~/
\end{lstlisting}
\FloatBarrier
\vspace{-5mm}

Alternatively and for the testbed administrator reference, Kodo can also
generate static libraries. We log in to the \ac{Raspi} and execute the unit
tests and one of the binaries by running:

\begin{lstlisting}[]
$ ssh pi@<RASP_IP>
$ ./kodocpp_tests
...
[----------] Global test environment tear-down
[==========] 17 tests from 15 test cases ran. (XXXX ms total)
[  PASSED  ] 17 tests.
$ ./encode_decode_simple
Data decoded correctly
\end{lstlisting}
\FloatBarrier
\vspace{-5mm}

If the Kodo cross-compilation worked properly, both the unit tests and
binary run should provide the shown outputs.

%\subsection{Kodo python}
%
%Often when simplicity and fast prototyping is of higher importance than speed
%it is often desired to use a scripting language rather than the often faster
%compiled language. Kodo-python provides roughly similar bindings to the
%underlying C++11 implementation of Kodo as Kodo-cpp does. This means that
%python calls to Kodo-python are actually executed in the underlying C++11
%compilied code.
%
%Parsing calls and variables between Python and C++ compiled code is
%not free, but this small introduced overhead may in many cases be worth
%it to take advantage of Pythons and other scripting languages many
%benefits.
%
%Below is a short procedure to use Kodo-python in a \ac{Raspi}.
%
%One advantage of Python compared to Kodo-cpp is that only the Kodo-python
%library needs to be cross compiled. The python scripts using the Kodo
%library needs not to be cross compiled but simply copied to the same
%directory as .
%
%Kodo-python is available at \url{https://github.com/steinwurf/kodo-python}.
%\begin{lstlisting}[]
%$ apt-get update
%$ apt-get install python build-essential libpython-dev python-dev
%\end{lstlisting}
%\FloatBarrier
%\vspace{-5mm}
%  % IN ARCH LINUX, MOD_PYTHON NEEDS TO BE INSTALLED
%
%
%Create a directory to hold our code:
%\begin{lstlisting}[]
%$ CODEDIR="${HOME}/code"
%$ mkdir -p ${CODEDIR}
%$ cd ${CODEDIR}
%\end{lstlisting}
%\FloatBarrier
%\vspace{-5mm}
%
%\begin{lstlisting}[]
%$ git clone git@github.com:steinwurf/kodo-python.git
%\end{lstlisting}
%\FloatBarrier
%\vspace{-5mm}
%
%\begin{lstlisting}[]
%$ cd kodo-python
%$ python2.7 waf configure --cxx_mkspec=cxx_raspberry_gxx49_arm
%$ python waf build
%\end{lstlisting}
%\FloatBarrier
%\vspace{-5mm}
%
%KODO-PYTHON IS NOT WORKING!! WAITING FOR HELP ON STEINWURF DEV.
%
%WRITE HOW TO RUN A KODO PYTHON SCRIPT IN THE RASPIS

%\texttt{hello\_world} returns immediately. This is not always the case.
%Sometimes it is desired to run simulations that should run for days.
%A software package called \texttt{screen} can be used for this purpose.
%Simply run ...
