
%\subsection{Kodo cross-compilation: From your PC to the Raspberry Pi}

Some processing expensive tasks such as compiling code libraries can with
advantage be performed in a \ac{PC} rather than a \ac{Raspi} to save time.
This section will present how to setup a toolchain in a \ac{PC} that may
be used to cross compile C++ code to run in \ac{Raspi}s.
There will be given an example of compiling a simple C++ application to run
on \ac{Raspi}, but also how Kodo-cpp can be cross compiled for \ac{Raspi}.

%Some tasks can with advantage be performed in \ac{PC} rather than a \ac{Raspi}
%due to the 
%
%Modern \ac{PC}s are significantly better equipped than the \ac{Raspi}s that in
%contrasts to \ac{PC}s target a far more affortable and smaller module size. This
%means that it may be preferred or even mandatory to perform some computational
%expensive tasks in a \ac{PC} rather than in \ac{Raspi}s.
%Example of tasks that are quite computational expensive is to compile software
%packages and large libraries.
%Another reason to use a \ac{PC} may be the hardware limitations of \ac{Raspi}s
%in terms of e.g. memory and disk space. Some software packages might simply be
%too memory demanding to be compiled in a \ac{Raspi}.
%Thus, this section will present how to setup a toolchain in a \ac{PC} that can
%be used to cross compile C++ code to the ARM architecture that the \ac{Raspi}s.
%The toolchain is mandatory on most consumer \ac{PC}s due to the different
%processor architecture.
%
%Furthermore, we will cover how to cross compile a simple C++ example and
%Kodo-cpp as well as how the binaries can be copied from a \ac{PC} to a
%\ac{Raspi} and executed.
%There exists bindings to Kodo in various programming language. We will present
%how to also run a Kodo-python application in a \ac{Raspi}.

%WRITE INTRO OF WHY WE WANT TO CROSS COMPILE. E.G. 1) WE DO NOT HAVE MONITORS
%CONNECTED TO THE RASPIS, 2) A PC IS SIGNIFICANTLY FASTER AND FAR BETTER EQUIPPED (RAM, CPU, HARDDRIVE), 3) MORE CONVINIENT TO DO EVERYTHING IN ONE PC AND DISTRIBUTE IT TO THE RASPIS AFTERWARDS. MORE?

\subsection{Setup toolchain on PC}

%Besides the previous description (\textbf{Include compiling Kodo from the
%RasPi from the scratch}), the testbed administrator can compile Kodo in its
%personal workstation and transfer the generated binaries directly to
%a path in the \ac{Raspi}. To achieve this, we get a toolchain that
%contains the binaries for the \texttt{raspberry-gxx49-arm-g++} compiler
%for the \ac{Raspi}. Therefore, we strongly recommend any testbed
%administrator to do the following procedure. In what follows, we provide
%the instructions considering that the NFS server uses the \texttt{\$HOME}
%directory as the working directory. However, the administrator may choose
%some other working directory of its preference if desired.

\begin{enumerate}

\item Create toolchain directory:
\begin{lstlisting}[]
$ TOOLCHAINDIR="${HOME}/toolchains"
$ mkdir -p $TOOLCHAINDIR
$ cd $TOOLCHAINDIR
\end{lstlisting}
\FloatBarrier
\vspace{-5mm}

\item Download the \ac{Raspi} toolchain for 64-bit Linux from.
A more recent \ac{Raspi} toolchain may be available at
\texttt{http://buildbot.steinwurf.dk/toolchains/linux/}:

\begin{lstlisting}[]
$ TOOLCHAIN="raspberry-gxx49-arm"
$ wget http://kom.aau.dk/project/TuneSCode/raspi/${TOOLCHAIN}.zip
\end{lstlisting}
\FloatBarrier
\vspace{-5mm}

\item Extract the downloaded file:
%locally in the NFS server. After
%this operation, there should be a new directory for the toolchain
%in the server. \\
\begin{lstlisting}[]
$ unzip ${TOOLCHAIN}.zip
\end{lstlisting}
\FloatBarrier
\vspace{-5mm}

\item (Optional) Delete zip file:
%locally in the NFS server. After
%this operation, there should be a new directory for the toolchain
%in the server. \\
\begin{lstlisting}[]
$ rm ${TOOLCHAIN}.zip
\end{lstlisting}
\FloatBarrier
\vspace{-5mm}

\item We can now verify that the ARM cross compiler is working:

\begin{lstlisting}[]
$ ${TOOLCHAINDIR}/${TOOLCHAIN}/bin/${TOOLCHAIN}-g++ --version
raspberry-gxx49-arm-g++ (crosstool-NG 1.21.0) 4.9.3 20150311 (prerelease)
Copyright (C) 2014 Free Software Foundation, Inc.
This is free software; see the source for copying conditions.  There is NO
warranty; not even for MERCHANTABILITY or FITNESS FOR A PARTICULAR PURPOSE.
\end{lstlisting}
\FloatBarrier
\vspace{-5mm}

%\begin{lstlisting}[]
%$ echo "${TOOLCHAINDIR}/${TOOLCHAIN}/bin/${TOOLCHAIN}-g++"
%/home/<USER>/toolchains/raspberry-gxx49-arm/bin/raspberry-gxx49-arm-g++
%\end{lstlisting}
%\FloatBarrier
%\vspace{-5mm}
    
%    The ARM cross compilers should now be located in

\item Make toolchain binaries available systemwide. Instead of calling
the ARM cross compiler using its full path, we can make it accessible
from the command shell systemwide. One way to do this is by adding the 
toolchain binaries directory to the Linux environment variable \texttt{PATH}
when the \ac{OS} starts up:

%\item Add toolchain binaries to \texttt{PATH}. Instead specifying the full
%path to the toolchain binaries we can instead tell the operation system
%where to search for it. This makes the toolchain binaries available
%systemwide.

\begin{lstlisting}[]
$ echo PATH=\"\$PATH:${TOOLCHAINDIR}/${TOOLCHAIN}/bin\" >> ${HOME}/.profile
\end{lstlisting}
\FloatBarrier
\vspace{-5mm}
%$ printf 'PATH="%s/%s/bin"\n' "${TOOLCHAINDIR}" "${TOOLCHAIN}" >> ${HOME}/.profile

%\item Add the \texttt{bin} folder of the toolchain to the \texttt{PATH}
%Linux environment variable of the server. This will help the server OS
%to recognize the location of the compiler command, which will be needed
%later. To do so, edit the \texttt{\$HOME/.profile} to add in a newline:
%\texttt{PATH="\$PATH:\$HOME/raspberry-gxx49-arm/bin"}. Save the
%\texttt{\$HOME/.profile}. \\

\texttt{.profile} should now contain the line we inserted. There may be more
code in you file.
% MAC and Hostname file
\Suppressnumber\begin{lstlisting}[]
<@\textcolor{gray}{\$HOME/.profile}@>
<@\textcolor{gray}{
---------------------------------------------------------------}
\Reactivatenumber @>
PATH="$PATH:/home/<USERNAME>/toolchains/raspberry-gxx49-arm/bin"
\end{lstlisting}
\FloatBarrier
\vspace{-5mm}

\item Update \texttt{PATH}. Source \texttt{.profile} to make the changes
take effect in your system:
\begin{lstlisting}[]
$ source ${HOME}/.profile
\end{lstlisting}
\FloatBarrier
\vspace{-5mm}

\item ARM cross compiler should now be available systemwide:

\begin{lstlisting}[]
$ raspberry-gxx49-arm-g++ --version
\end{lstlisting}
\FloatBarrier
\vspace{-5mm}

%\item Restart the server session in order for the changes made in the
%previous step take effect. To verify this, open a new terminal and type:
%\texttt{raspberry-gxx49-arm-g++ --version}. A correct binary installation
%should return an output similar to:
%
%\texttt{raspberry-gxx49-arm-g++ (crosstool-NG 1.21.0) 4.9.3 20150311 (prerelease)
%Copyright (C) 2014 Free Software Foundation, Inc.
%This is free software; see the source for copying conditions.  There is NO
%warranty; not even for MERCHANTABILITY or FITNESS FOR A PARTICULAR PURPOSE.} \\

\end{enumerate}

\subsection{Cross compile example}

The following example will provide a very basic example of 1) how to cross compile
a \texttt{hello\_world} c++ example to the \ac{Raspi}'s ARM architecture
and 2) how to copy and execute the binary in a \ac{Raspi} using \ac{SCP} and
\ac{SSH}.

Lets create a directory to hold our code:
\begin{lstlisting}[]
$ CODEDIR="${HOME}/code"
$ mkdir -p ${CODEDIR}
$ cd ${CODEDIR}
\end{lstlisting}
\FloatBarrier
\vspace{-5mm}

Create the file \texttt{hello\_world.cpp} with the following content using your
favorite text editor:

\Suppressnumber\begin{lstlisting}[]
<@\textcolor{gray}{\${CODEDIR}/hello\_world.cpp}@>
<@\textcolor{gray}{
---------------------------------------------------------------}
\Reactivatenumber @>
#include <iostream>

int main()
{
    std::cout << "Hello World!" << std::endl;
    return 0;
}
\end{lstlisting}
\FloatBarrier

Save \texttt{hello\_world.cpp} and compile it for \ac{Raspi}:

\begin{lstlisting}[]
$ raspberry-gxx49-arm-g++ hello_world.cpp -o hello_world
\end{lstlisting}
\FloatBarrier
\vspace{-5mm}

This should produce a binary \texttt{hello\_world}. Copy it to a \ac{Raspi}
and test it. We will use \ac{SCP} to copy the binary file to one of the \ac{Raspi}s.

The default username for our Raspbian lite distribution is "pi" and password is "raspberry"

\begin{lstlisting}[]
$ scp main pi:<RASP_IP>:~/
\end{lstlisting}
\FloatBarrier
\vspace{-5mm}

If you do not know the \ac{IP} address of a \ac{Raspi} in your network, you can
connect a monitor to it and run the following command after you have logged in:
\begin{lstlisting}[]
pi@rasp01:~ $ ifconfig
eth0      Link encap:Ethernet  HWaddr b8:27:eb:72:77:54  
          inet addr:192.168.87.107  Bcast:192.168.87.255  Mask:255.255.255.0
          inet6 addr: fe80::e0a5:38f3:6f82:bc79/64 Scope:Link
          UP BROADCAST RUNNING MULTICAST  MTU:1500  Metric:1
          RX packets:1537 errors:0 dropped:0 overruns:0 frame:0
          TX packets:445 errors:0 dropped:0 overruns:0 carrier:0
          collisions:0 txqueuelen:1000 
          RX bytes:259117 (253.0 KiB)  TX bytes:52551 (51.3 KiB)

\end{lstlisting}
\FloatBarrier
\vspace{-5mm}

After the executable has been copied to a \ac{Raspi}. Then, \ac{SSH} to it:

\begin{lstlisting}[]
$ ssh pi:<RASP_IP>
\end{lstlisting}
\FloatBarrier
\vspace{-5mm}

We can list the directory content after we have logged into the \ac{Raspi} and
see that \texttt{hello\_world} is there:

\begin{lstlisting}[]
pi@rasp07:~ $ ls
hello_world  rasp_config
\end{lstlisting}
\FloatBarrier
\vspace{-5mm}

Now, simply execute the \texttt{hello\_world} to confirm that cross compiling
\texttt{hello\_world} worked properly:

\begin{lstlisting}[]
pi@rasp07:~ $ ./hello_world
Hello World!
\end{lstlisting}
\FloatBarrier
\vspace{-5mm}

\subsection{Cross compile Kodo}

Kodo is a C++11 network coding library. It is available through bindings in
a variety of programming languages.
%There exists a number of bindings for
%Kodo to other popular programming languages.
This procedure will present how to configure the Kodo C++ bindings (Kodo-cpp)
to cross compile applications that can run in \ac{Raspi}. More details are
provided in the code documentation~\cite{kodocppdoc}.

%This procedure will go through
%how to setup Kodo-cpp to cross compile applications that can run in \ac{Raspi}. Kodo-cpp provides a
%simple interface to the underlying C++11 code that exists in the libraries
%Kodo-core and Kodo-rlnc.

To use Kodo, it is required to obtain a research- and
education-friendly license.
It can be be obtained from~\cite{steinwurflicenselink} and is free
for research and educational purposes.

%A more detailed procedure of setting up Kodo-cpp than provided below is available at

\begin{enumerate}

\item Start by creating a directory to hold the code:
\begin{lstlisting}[]
$ CODEDIR="${HOME}/code"
$ mkdir -p ${CODEDIR}
$ cd ${CODEDIR}
\end{lstlisting}
\FloatBarrier
\vspace{-5mm}

\item Clone the Kodo repository and change directory into the repository:
\begin{lstlisting}[]
$ git clone https://github.com/steinwurf/kodo-cpp.git
$ cd kodo-cpp
\end{lstlisting}
\FloatBarrier
\vspace{-5mm}

\item Configure Kodo to build executables for the \ac{ARM} architecture using the \ac{Raspi} toolchain:
\begin{lstlisting}[]
$ python waf configure --cxx_mkspec=cxx_raspberry_gxx49_arm
...
'configure' finished successfully (0.620s)
\end{lstlisting}
\FloatBarrier
\vspace{-5mm}

\item Build executables:
\begin{lstlisting}[]
$ python waf build
...
'build' finished successfully (2m22.918s)
\end{lstlisting}
\FloatBarrier
\vspace{-5mm}


\item Make shared library:
\begin{lstlisting}[]
$ python waf install --install_shared_libs --install_path="./shared_test"
\end{lstlisting}
\FloatBarrier
\vspace{-5mm}

\item Copy shared library to a \ac{Raspi}'s home directory (Alternatively, Kodo can also generate static libraries):
\begin{lstlisting}[]
$ scp -r shared_test/include shared_test/libkodoc.so pi@<RASP_IP>:~/
\end{lstlisting}
\FloatBarrier
\vspace{-5mm}

\item Copy a binary using Kodo shared library to the \ac{Raspi}'s home directory:
\begin{lstlisting}[]
$ scp -r shared_test/encode_decode_simple pi@<RASP_IP>:~/
\end{lstlisting}
\FloatBarrier
\vspace{-5mm}

\item Execute binary:
\begin{lstlisting}[]
$ ssh pi@<RASP_IP> ./encode_decode_simple
Data decoded correctly
\end{lstlisting}
\FloatBarrier
\vspace{-5mm}

Cross compiling Kodo applications works provided that above command
returned "Data decoded correctly".

%\item Navigate to the repository and configure \texttt{waf} by typing:
%\texttt{python config.py} and select the 16th ``make specification'' file
%for the \ac{Raspi}, e.g. option \texttt{[16]cxx\_raspberry\_gxx49\_arm}
%presented by the file.
%
%This command configures \texttt{waf} to use the proper compiler and its
%required flags to generate the binaries for the \ac{Raspi}. If the
%configuration was correct, the output will indicate:
%\texttt{'configure' finished successfully (X.XXXs)}, where \texttt{X.XXX}
%is total time in seconds for configuring the project in the server. \\
%
%\item Execute \texttt{python waf build}. If the build process was
%successful, the generated binaries for the \ac{Raspi} should be located
%in \texttt{build/cxx\_raspberry\_gxx49\_arm} in the Kodo repository.
%\textbf{Indicate how the binary files should look like}.
%
%Once this procedure is made, the testbed administrator can relocate the
%generated binary files to the \ac{Raspi}s through the network as desired
%by using the \texttt{scp} command during the configuration step.


\end{enumerate}

%\subsection{Kodo Builds for the \ac{Raspi}, Platform Support and Documentation \textcolor{red}{(WHAT IS THE PURPOSE OF THIS SUBSECTION?)}}
%
%You can check the build status of Kodo, Fifi and other relevant projects
%through their respective repository master branch on our buildbot page
%\cite{steinwurf2016buildbot}. Our buildbot displays the status of the builds
%for Raspbian 8 and GCC 4.9 for the ARM architecture which is the relevant one
%for the \ac{Raspi}. At the link, you can check build status and build
%statistics. Also, documentation about Kodo basics with a tutorial is available
%at \cite{kododocs}.






%\subsection{Kodo python}
%
%Often when simplicity and fast prototyping is of higher importance than speed
%it is often desired to use a scripting language rather than the often faster
%compiled language. Kodo-python provides roughly similar bindings to the
%underlying C++11 implementation of Kodo as Kodo-cpp does. This means that
%python calls to Kodo-python are actually executed in the underlying C++11
%compilied code.
%
%Parsing calls and variables between Python and C++ compiled code is
%not free, but this small introduced overhead may in many cases be worth
%it to take advantage of Pythons and other scripting languages many
%benefits.
%
%Below is a short procedure to use Kodo-python in a \ac{Raspi}.
%
%One advantage of Python compared to Kodo-cpp is that only the Kodo-python
%library needs to be cross compiled. The python scripts using the Kodo
%library needs not to be cross compiled but simply copied to the same
%directory as .
%
%Kodo-python is available at \url{https://github.com/steinwurf/kodo-python}.
%
%
%
%\begin{lstlisting}[]
%$ apt-get update
%$ apt-get install python build-essential libpython-dev python-dev
%\end{lstlisting}
%\FloatBarrier
%\vspace{-5mm}
%  % IN ARCH LINUX, MOD_PYTHON NEEDS TO BE INSTALLED
%
%
%Create a directory to hold our code:
%\begin{lstlisting}[]
%$ CODEDIR="${HOME}/code"
%$ mkdir -p ${CODEDIR}
%$ cd ${CODEDIR}
%\end{lstlisting}
%\FloatBarrier
%\vspace{-5mm}
%
%\begin{lstlisting}[]
%$ git clone git@github.com:steinwurf/kodo-python.git 
%\end{lstlisting}
%\FloatBarrier
%\vspace{-5mm}
%
%\begin{lstlisting}[]
%$ cd kodo-python
%$ python2.7 waf configure --cxx_mkspec=cxx_raspberry_gxx49_arm
%$ python waf build
%\end{lstlisting}
%\FloatBarrier
%\vspace{-5mm}
%
%KODO-PYTHON IS NOT WORKING!! WAITING FOR HELP ON STEINWURF DEV.
%
%WRITE HOW TO RUN A KODO PYTHON SCRIPT IN THE RASPIS

\subsection{Fabric}

Controlling multiple devices using \ac{SSH} from a single \ac{PC}
often leads to a lot of repetitive work. E.g.

\begin{itemize}
    \item rebooting a set of devices
    \item installing applications in multiple devices
    \item copy files to/from multiple devices
    %\item run a script in multiple devices
\end{itemize}

\texttt{Fabric} provides a python library to ease the task of working
with multiple devices from a single \ac{PC}. The code below provides
a small script that can perform the few items above:



% Exit chroot, umount, and write to memory card
\Suppressnumber\begin{lstlisting}[]
<@\textcolor{gray}{\$CODEDIR/fabfile.py}@>
<@\textcolor{gray}{
-------------------------------------------}
\Reactivatenumber @>
from fabric.api import run, env, task
# Python Fabric script to run commands on multiple hosts through ssh
#
# Run script as 'fab <task>', where <task> is one of the scripts functions
# marked as a tesk. The task marked as 'default' will be run if <task> is not
# specified

env.hosts = ['rasp00.domain.com','rasp01.domain.com','rasp02.domain.com']
env.user = 'pi'
env.password = 'raspberry'

@task
def reboot():
    """ Reboot device """
    sudo('reboot', quiet=True)

@task
def install(program):
    """
    Install a program
    program: program name
    """
    result = sudo('apt-get install -y {}'.format(program), quiet=True)
    print(result)

@task
def push(src,dst):
    """
    Copy file to device
    src: source file path
    dst: destination file path
    """
    put(src, dst)

\end{lstlisting}
\FloatBarrier

The script functions can be called from the shell in the following
manner. \texttt{Fabric} will automatically search the directory for a file 
named 'fabfile.py'. The syntax for executing a task and arguments is in the form
'fab <FUNCTION>:arg1,arg2,...:
\begin{lstlisting}[]
$ fab reboot
$ fab install:tmux
\end{lstlisting}
\FloatBarrier
\vspace{-5mm}
The first function above reboots the \ac{Raspi}s in the lists of hosts, and
the install function installs a program given by an argument. In the above case,
\texttt{tmux} is installed.
%The two functions above reboots the \ac{Raspi}s specified in the list of hosts

Below is a slightly more complicated function call that copies a file from 
a \ac{PC} to all \ac{Raspi}s in the list of hosts:

\begin{lstlisting}[]
$ fab push:"${CODEDIR}/hello_world",'~/'
[192.168.87.107] Executing task 'push'
[192.168.87.107] put: /home/<USER>/code/hello_world -> /home/pi/hello_world

Done.
Disconnecting from 192.168.87.107... done.
\end{lstlisting}
\FloatBarrier
\vspace{-5mm}

\texttt{Fabric} has many other useful functionalities that are useful in
controlling a bunch of \ac{Raspi}s.

%execute a script function by calling "fab install:feh"




\subsection{Long running jobs using SSH}

There may be times when a simulation has to run for days in the \ac{Raspi}s.
For this purpose, it may be inconvenient to keep a \ac{SSH} connection
open to the \ac{Raspi}s risking that the connection will be interrupted and
thus leading to the \ac{Raspi} automatically terminating the process.

%There may be times when it is desired to start a simulation that will
%run for days. For that purpose, it may not be desired to keep
%a \ac{SSH} connection open to the \ac{Raspi} risking that the connection
%will be interrupted and thus result in the \ac{Raspi} terminating
%the running simulation.

There exists a few methods to enable the \ac{Raspi}s to continue running
applications although the connection is terminated either on purpose or
unexpectedly.
%There exists a few methods to avoid the risk of simulations terminating
%unexpectedly.
One method is to run programs within a \texttt{screen} session.
\texttt{screen} enables a user to run applications within a \texttt{screen}
session which never terminates. Users can attach and detach from such a
session as desired.
%a session as it is
%desired. Programs within the screen session just keeps running.

The following procedure presents how to use \texttt{screen} with \ac{SSH}:
\begin{enumerate}
    \item login to a \ac{Raspi}
    \item open a screen session
    \item execute a command
    \item detach from the screen session
    \item terminate SSH connection
    \item login to the \ac{Raspi} again
    \item attach to screen session to see the program still running
\end{enumerate}

Start by establishing SSH connection to a \ac{Raspi} and open a screen session:
\begin{lstlisting}[]
$ SSH pi@<RASP_IP>
$ screen
Screen version 4.02.01 (GNU) 28-Apr-14
...
\end{lstlisting}
\FloatBarrier
\vspace{-5mm}
Press space or return to clear the shell.

At this stage, we are in a screen session. Now execute a program that never ends:
\begin{lstlisting}[]
$ top
\end{lstlisting}
\FloatBarrier
\vspace{-5mm}

When \texttt{top} is running, press Ctrl+a Crtl+d to detach from
the \texttt{screen} session.

Terminate the SSH connection and login again:
\begin{lstlisting}[]
$ exit
$ SSH pi@<RASP_IP>
\end{lstlisting}
\FloatBarrier
\vspace{-5mm}

Now that we are logged in to the \ac{Raspi} again, lets see if a
\texttt{screen} session is still running:
\begin{lstlisting}[]
$ screen -list
There is a screen on:
	1668.pts-0.rasp09	(06/06/16 04:02:15)	(Detached)
1 Socket in /var/run/screen/S-pi.
\end{lstlisting}
\FloatBarrier
\vspace{-5mm}

It is seen that the session is still running and that no user is currently
attached to the session. Attach to the session:
\begin{lstlisting}[]
$ screen -r 1668.pts-0.rasp09
\end{lstlisting}
\FloatBarrier
\vspace{-5mm}

After attaching again, \texttt{top} should still be running. \texttt{screen}
has a few more functionalities that is outside the scope of this
paper.

To terminate the \texttt{screen} session. First, terminate \texttt{top}
by pressing 'q', and then type 'exit' two times to first exit the
\texttt{screen} session and then terminate the \ac{SSH} connection:


%Problem when you logout, then applications will terminate. SCREEN is the answer.

%\texttt{hello\_world} returns immediately. This is not always the case.
%Sometimes it is desired to run simulations that should run for days. 
%A software package called \texttt{screen} can be used for this purpose.
%Simply run ...
