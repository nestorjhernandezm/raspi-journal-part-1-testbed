\label{sec:image_setup}
In this section, we review the steps to create a common \ac{OS} image for
all the \ac{Raspi}s. The image setup is composed of three major steps:
Select and download the \ac{OS} image file, alter the image structure and
configure the \ac{OS} files. We proceed to detail all these steps providing
brief discussions to our setup choices when required. To perform these
steps, we indicate with command-line blocks the required sequential
commands to be typed by the testbed administrator on his / her \ac{PC}
to obtain the desired setting. In all the command blocks in the paper, we indicate if a command needs to be run with root permissions (\#) or common user permissions (\$). These signs will prefix the commands.

\subsection{OS Selection and Download}

To get started, we first need to install an \ac{OS} that works properly
on all the \ac{Raspi} models. We will download and setup the image in
the testbed administrator \ac{PC} using a Debian-based distribution.
An alternative to this method, is to create a tailoared Linux distribution
for the \ac{Raspi} platform using the Yocto Project \cite{yocto}. However, this
process would require to assemble and compile all the software for the
\ac{Raspi} platform from scratch which goes beyond of the scope of our
work. We use the popular Debian-based Raspbian Linux~\cite{raspbian} given
that is the recommended and default \ac{OS} for the \ac{Raspi}. Raspbian is
made available in two bundles: Raspbian and Raspbian Lite. The difference
between the two, is that Raspbian contains a pre-installed desktop environment
for user interaction and Raspbian Lite by default only permits to interact
through a command shell. Given that the \ac{Raspi}s in our testbed are not
connected to monitors, we decide to work with Raspbian Lite.
If required, a desktop environment can be installed using the package manager later.
%\todot{remove the rest of this line} as it will be shown in the configuration section.

% The overall procedure to customize the official Raspbian Lite image
% are:
% \begin{enumerate}
%     \item Download Raspbian Lite
%     \item Alter Raspbian Lite. e.g. browse, modify, add, and delete files
%         in the official Raspbian Lite image
%     \item Change root, i.e. change root filesystem into the Raspbian Lite
%         image to update and install software packages
%     \item Write image to memory cards
% \end{enumerate}

%\subsection{Download Raspbian Lite}
%To download the latest Raspbian Lite, we go to the url
%\url{http://downloads.raspberrypi.org/raspbian_Lite/images/}.

The latest Raspbian Lite bundle can be downloaded from the Raspbian
official webpage~\cite{raspbian}. At the time of this writing, the latest
available bundle was \texttt{2016-05-27-raspbian-jessie-lite.zip}.
To ensure that the content of the bundle does not change, this procedure
is based on that particular version of Raspbian Lite which we have
made available at~\cite{soerensen_chres_wiant_2016_154143}. All other files used in this paper are also available there. The testbed
administrator has to move these files to his / her own \ac{HTTP} server.
To get started, the testbed administrator must open a Linux
shell (terminal) on his / her \ac{PC} and declare the environment variables shown
in the command block below. We show the whole procedure by performing the role
of the testbed administrator.
%To copy the commands from this procedure to the
%testbed administrator terminal, we recommend to read this document
%with Adobe Acrobat Reader in Linux. Otherwise, copied characters might not
%render properly when pasted.

%After the terminal has been opened, we start by declaring a few variables to reduce
%repeated typing. The first variables we declare is the Raspbian image name.
%Notice that the extension has been omitted. This is because the image has been
%packed into a zip file. The other variable we declare is a working directory.
%This is where we will download the image to and work on it. In other words, it
%will be stored in \texttt{/home/<username>/Raspbian}

%To download Raspbian lite, we go to the \url{http://downloads.raspberrypi.org}.
%There should be a folder called raspbian\lite and
%to find the download page at which Raspbian lite should be located. Instead of
%downloading with the browser, we just extract the download url to the latest
%image. At the time of this writing, that was

% Define variable]
\begin{lstlisting}[]
$ export URL="https://zenodo.org/record/154328/files/"
$ export IMAGE="2016-05-27-raspbian-jessie-lite"
$ export WORKDIR="${HOME}/Raspbian"
\end{lstlisting}
\FloatBarrier
\vspace{-5mm}

In this code block, the \texttt{\$\{URL\}} and \texttt{\$\{IMAGE\}}
variables specify where the Linux bundle is located and
\texttt{\$\{WORKDIR\}} specifies a working directory where the Raspbian
Lite bundle will be downloaded and customized. If the testbed administrator
allocates his / her files into another location, then it will be required to
change the \texttt{\$\{URL\}} environment variable. Notice that even though we
use the \$ and \# signs in the shell, in general these signs will be
particular to the testbed administrator \ac{OS} shell. Next, we create
the working directory and change to it with the \texttt{cd} command. To
download the image, we utilize the \texttt{wget} command before unpacking
the \texttt{zip} file as follows:

% Running a command as root can on most systems be done by putting
% \texttt{sudo} in front of the command. This is illustrated in the
% following code block
% with the command \texttt{whoami} that prints the username. Lines within a code
% block without leading \$ or \# is terminal output or content of a file.

% Root example
% \begin{lstlisting}[]
% $ whoami
% <USERNAME>
% $ sudo whoami
% root
% \end{lstlisting}
% \FloatBarrier
% \vspace{-5mm}

% Download and unpack image
\begin{lstlisting}[]
$ mkdir -p ${WORKDIR}
$ cd ${WORKDIR}
$ wget ${URL%/}/${IMAGE}.zip
$ unzip ${IMAGE}.zip
\end{lstlisting}
\FloatBarrier
\vspace{-5mm}

%A more recent version of Raspbian lite may be available at~\cite{raspbian}.
%\url{http://downloads.raspberrypi.org/raspbian_lite/images}.

\subsection{Image Customization}

After Raspbian Lite has been unpacked, there should be an \texttt{.img}
file in the working directory \texttt{\$\{WORKDIR\}}. \texttt{fdisk} can
be used to display the content of the image. We parse the arguments \texttt{-u sectors}
to display the sizes in sectors and \texttt{-l} to display the partitions
within the image. The \texttt{fdisk} command should output to the terminal
something similar to:

% Check out the image
% The dollar hack was to fix vim syntax
\begin{lstlisting}[literate={DOLLAR}{\$}1]
DOLLAR fdisk -u=sectors -l ${IMAGE}.img
Disk 2016-05-27-raspbian-jessie-lite.img: 1.3 GiB, 1387266048 bytes, 2709504 sectors
Units: sectors of 1 * 512 = 512 bytes
Sector size (logical/physical): 512 bytes / 512 bytes
I/O size (minimum/optimal): 512 bytes / 512 bytes
Disklabel type: dos
Disk identifier: 0x6fcf21f3

Device                               Boot  Start     End Sectors  Size Id Type
2016-05-27-raspbian-jessie-lite.img1        8192  137215  129024   63M  c W95 FAT32 (LBA)
2016-05-27-raspbian-jessie-lite.img2      137216 2709503 2572288  1.2G 83 Linux
\end{lstlisting}
\FloatBarrier
\vspace{-5mm}

The output provides relevant information about the image. The image is in
total 2709504 sectors (1.3 GiB) in size and contains two partitions. The
first partition starts at sector 8192 and the other partition starts at
sector 137216. The first partition type is FAT32 with a size of 63 MB
and the second partition is of type Linux with a size of 1.2 GB. This
indicates that the first partition is a boot partition and the second
one is a traditional Linux filesystem. In this case, the root filesystem, i.e. \texttt{/}.

\subsection{Image Resizing}
Given that we want to customize the root filesystem in the \ac{Raspi}s,
we need to expand the image file since 1.2 GB might not be enough to store
the existing root filesystem plus additional files and software packages.
%Given that we want to customize the files stored in the \ac{Raspi}s,
%we need to resize the image file since 1.2 GB might not be enough to store
%the root filesystem with the additional packages that will be installed in this procedure.
%due to the total size of the additional packages.
Thus,
we need to increase the partition size. The following procedure illustrates
how the image and its root filesystem can be expanded by one GB.
First, to expand the image one GB, we execute:

\begin{lstlisting}[]
$ dd if=/dev/zero bs=1M count=1024 >> ${IMAGE}.img && sync
\end{lstlisting}
\FloatBarrier
\vspace{-5mm}

Later, we use \texttt{fdisk} with the same arguments as before to see that
the image is now one GB larger:
\begin{lstlisting}[literate={DOLLAR}{\$}1]
DOLLAR fdisk -u=sectors -l ${IMAGE}.img
Disk 2016-05-27-raspbian-jessie-lite.img: 2.3 GiB, 2461007872 bytes, 4806656 sectors
Units: sectors of 1 * 512 = 512 bytes
Sector size (logical/physical): 512 bytes / 512 bytes
I/O size (minimum/optimal): 512 bytes / 512 bytes
Disklabel type: dos
Disk identifier: 0x6fcf21f3

Device                               Boot  Start     End Sectors  Size Id Type
2016-05-27-raspbian-jessie-lite.img1        8192  137215  129024   63M  c W95 FAT32 (LBA)
2016-05-27-raspbian-jessie-lite.img2      137216 2709503 2572288  1.2G 83 Linux
\end{lstlisting}
\FloatBarrier
\vspace{-5mm}

Now, in the above command block output, we observe that the change has taken
effect by noticing the total available image size is 2.3 GiB. To expand the
root filesystem, we replace the Linux partition with a new partition one GB
larger. The starting point of this new partition should be the same as the old
one. We make use of \texttt{fdisk} to alter the partition table
in the commands below. They (i) delete partition number 2, (ii) create a new
primary partition and (iii) set the new partition starting point. The new
partition starting point value is 137216 in our case. Finally, we (iv) write
the new partition table to the image file. This is made as follows:

%\begin{lstlisting}[]
%$ (echo d; echo 2; echo n; echo ; echo ; echo 137216; echo ; echo w) | fdisk ${IMAGE}.img
%\end{lstlisting}
\begin{lstlisting}[]
$ fdisk ${IMAGE}.img << EOF
d
2
n
p
2
137216

w
EOF
\end{lstlisting}
\FloatBarrier
\vspace{-5mm}

If the partitions commands were correct, the partition table should now
look like the following:
\begin{lstlisting}[literate={DOLLAR}{\$}1]
DOLLAR fdisk -u=sectors -l ${IMAGE}.img
Disk 2016-05-27-raspbian-jessie-lite.img: 2.3 GiB, 2461007872 bytes, 4806656 sectors
Units: sectors of 1 * 512 = 512 bytes
Sector size (logical/physical): 512 bytes / 512 bytes
I/O size (minimum/optimal): 512 bytes / 512 bytes
Disklabel type: dos
Disk identifier: 0x6fcf21f3

Device                               Boot  Start     End Sectors  Size Id Type
2016-05-27-raspbian-jessie-lite.img1        8192  137215  129024   63M  c W95 FAT32 (LBA)
2016-05-27-raspbian-jessie-lite.img2      137216 4806655 4669440  2.2G 83 Linux
\end{lstlisting}
\FloatBarrier
\vspace{-5mm}

% \begin{lstlisting}[]
% Command (m for help): Partition number (1-4):
% Command (m for help): Partition type:
%    p   primary (1 primary, 0 extended, 3 free)
%    e   extended
% Select (default p): Using default response p
% Partition number (1-4, default 2): Using default value 2
% First sector (2048-4806655, default 2048): Last sector, +sectors or
% +size{K,M,G} (137216-4806655, default 4806655): Using default value 4806655

% Command (m for help): The partition table has been altered!

% Syncing disks.

% \end{lstlisting}
% \FloatBarrier
% \vspace{-5mm}

\subsection{Loopback Device Setup}
After successfully resizing the image file, we use a loopback device to make
the Raspbian image available as a block device in the filesystem. For this
command to work, the testbed administrator distribution must have the
\texttt{util-linux} package with version 2.21 or higher. Otherwise, the
\texttt{-P} argument of \texttt{losetup} will appear as invalid. If the
version of \texttt{losetup} can not be updated for some reason, an
alternative option for this part is presented in
Section~\ref{sec:alternative_losetup} of the Appendices.

\begin{lstlisting}[]
$ export DEV=$(sudo losetup --show -f -P ${IMAGE}.img); echo $DEV
/dev/loop0
\end{lstlisting}
\FloatBarrier
\vspace{-5mm}

If the previous command was succesful, the \texttt{lsblk} command can be used
to list the available block devices in the filesystem as follows:
%Use \texttt{lsblk} to view the partitions:

\begin{lstlisting}[]
# lsblk
NAME      MAJ:MIN RM  SIZE RO TYPE MOUNTPOINT
...
loop0       7:0    0  2.3G  0 loop
|-loop0p1 259:2    0   63M  0 loop
|-loop0p2 259:3    0  2.2G  0 loop
...
\end{lstlisting}
\FloatBarrier
\vspace{-5mm}

The image block device appears as \texttt{/dev/loop0}. This block device has
two partitions associated to it, e.g. \texttt{loop0p1} and \texttt{loop0p2}.
Finally, we check the filesystem of the block device with \texttt{e2fsck} and
resize it with the \texttt{resize2fs} command:%, respectively in the command block:

\begin{lstlisting}[]
# e2fsck -f ${DEV}p2
e2fsck 1.42.8 (20-Jun-2013)
Pass 1: Checking inodes, blocks, and sizes
Pass 2: Checking directory structure
Pass 3: Checking directory connectivity
Pass 4: Checking reference counts
Pass 5: Checking group summary information
/dev/loop0p2: 35392/80480 files (0.1% non-contiguous), 201968/321536 blocks
# resize2fs ${DEV}p2
resize2fs 1.42.8 (20-Jun-2013)
Resizing the filesystem on /dev/loop0 to 583680 (4k) blocks.
The filesystem on /dev/loop0 is now 583680 blocks long.
\end{lstlisting}
\FloatBarrier
\vspace{-5mm}

\subsection{Block Devices Mounting}
For browsing and altering the files in the image, we mount the block
device partitions into a particular path of our \texttt{\$\{WORKDIR\}} in order
to customize them. We mount the block device partition that
contains the root filesystem and later the boot partition. This is done by
creating an empty directory that is used as a mountpoint. We name it
\texttt{root} and create it in the working directory before mounting the
root filesystem onto the mountpoint. We mount the root filesystem
as follows:

\begin{lstlisting}[]
$ export ROOTDIR="${WORKDIR}/root"
$ mkdir -p ${ROOTDIR}
# mount ${DEV}p2 ${ROOTDIR}
\end{lstlisting}
\FloatBarrier
\vspace{-5mm}
%# mount -o loop,offset=$((137216*512)) ${IMAGE}.img ${ROOTDIR}

The root filesystem mounted in \texttt{\$\{ROOTDIR\}}, has already a boot
directory that can be used as the mount point for the boot partition
in the block device \texttt{/dev/loop0p1}. This is convenient because
the final edited partition from \texttt{\$\{ROOTDIR\}/boot} will be mounted on
this same directory when a \ac{Raspi} starts up with a memory card
containing the raw final image. Hence, to mount boot partition we do:
%It is therefore the natural place to mount it for later purposes.

\begin{lstlisting}[]
# mount ${DEV}p1 ${ROOTDIR}/boot
\end{lstlisting}
\FloatBarrier
\vspace{-5mm}
%# mount -o loop,offset=$((8192*512)) ${IMAGE}.img ${ROOTDIR}/boot
%We can now change all the files in the disk image as desired.

In this way, it is now possible to change all files within the Raspbian
image as desired by editing the files in \texttt{\$\{ROOTDIR\}}. We take
advantage of this to edit configuration files, append new files and even
update and install packages.

\subsection{Image OS Files and Configuration Scripts Setup}
In general, the \ac{Raspi}s should be setup as similar as possible. However,
some particularities exist to differentiate the devices in principle. Also,
scripts containing further configurations for the \ac{Raspi}s are desirable
to be distributed as part of the common image. Therefore, here we present
the steps to setup basic properties of the \ac{Raspi}s and distributing
configuration scripts to each of them through the image. For this, we first
indicate how to obtain and put our configuration scripts in the image. Later,
we describe the tasks performed by these configuration scripts. Finally, we
indicate how and in which order the scripts are executed to configure all the
devices. Any testbed administrator might modify or include other
tasks according to his / her needs as we will show.

\subsubsection{Image Default Configuration Scripts Download}
\label{sec:configuration_files_download}
In our case, we have our default configuration scripts stored in a
file \texttt{rasp\_config.zip} located in the same \ac{URL} of the
\ac{HTTP} server where the image was retrieved from, i.e. the one in
the environment variable \texttt{\$\{URL\}}. We first download
this compressed file with \texttt{wget} and extract it locally into our
Raspbian Lite image. These commands and the output of the last one are
shown as follows:

% Get the configuration files that are compressed in a remote server
\begin{lstlisting}[]
$ wget ${URL%/}/rasp_config.zip
$ unzip rasp_config.zip -d ${ROOTDIR}/home/pi/
Archive:  rasp_config.zip
   creating: ${ROOTDIR}/home/pi/rasp_config/
  inflating: ${ROOTDIR}/home/pi/rasp_config/nodes.csv
  inflating: ${ROOTDIR}/home/pi/rasp_config/set_hostname
  inflating: ${ROOTDIR}/home/pi/rasp_config/main
  inflating: ${ROOTDIR}/home/pi/rasp_config/update_rasp_config
\end{lstlisting}
\FloatBarrier
\vspace{-5mm}

The unzippped files are one configuration file and three configuration
scripts in the newly created \texttt{\$\{ROOTDIR\}/home/pi/rasp\_config/}
folder in the image. In what follows, we describe which features do we
require all the \ac{Raspi}s to have and how are they achieved with
these configuration scripts.

\subsubsection{Device Hostnames}
The hostname helps the user to physically distinguish the
devices from each other. In our case we require the devices in our
testbed to have different hostnames. We define the hostnames based
on the \ac{MAC} addresses of the \ac{Raspi}s wired Ethernet interface.

Prior to this stage, the \ac{MAC} address of a network card can be found
using the command \texttt{ifconfig} or \texttt{ip addr} on a given
\ac{Raspi}. We store the \ac{MAC} addresses and hostnames of the
\ac{Raspi}s in the configuration file
\texttt{\$\{ROOTDIR\}/home/pi/rasp\_config/nodes.csv}. A sample of our file
is shown as follows:

% MAC and Hostname file
\Suppressnumber\begin{lstlisting}[]
<@\textcolor{gray}{\$\{ROOTDIR\}/home/pi/rasp\_config/nodes.csv}@>
<@\textcolor{gray}{
--------------------------------------------------------------------------------}
\Reactivatenumber @>
# Ethernet MAC    Hostname
b8:27:eb:5b:da:20 rasp00
b8:27:eb:7b:c3:91 rasp01
b8:27:eb:54:9c:64 rasp02
b8:27:eb:95:bd:11 rasp03
...
\end{lstlisting}
\FloatBarrier
\vspace{-5mm}

The testbed administrator has to insert the \ac{MAC} addresses and hostnames
of his / her \ac{Raspi}s obtained previously in the format shown in the
configuration file. For each given \ac{Raspi}, there is a \ac{MAC} address and
the corresponding hostname. This file will be employed by the
\texttt{\$\{ROOTDIR\}/home/pi/rasp\_config/set\_hostname} \ac{Bash} script to
assign the hostname of each \ac{Raspi}. The script content is the following:

% Set hostname
\Suppressnumber\begin{lstlisting}[]
<@\textcolor{gray}{\$\{ROOTDIR\}/home/pi/rasp\_config/set\_hostname}@>
<@\textcolor{gray}{
-------------------------------------------------------------------------------------}
\Reactivatenumber @>
#!/usr/bin/env bash

script_path="$(dirname $(realpath $0))"
config_file=${script_path}/nodes.csv
mac=$(cat /sys/class/net/eth0/address)
old_hostname=$(hostname)
new_hostname=$(grep $mac $config_file | cut -f2 -d' ')

# Assign hostname found in nodes.csv
if [ ! -z ${new_hostname} ]; then
    echo ${new_hostname} > /etc/hostname
    hostname ${new_hostname}
    sed -i.old -e "s:${old_hostname}:${new_hostname}:g" /etc/hosts
fi
\end{lstlisting}
\FloatBarrier
\vspace{-5mm}

The script (in lines): (1) tells the system to interpret the script
using \ac{Bash}, (3-4) gets the path to the script itself and the list of
hostnames, (5) gets the \ac{MAC} address of the node itself, (6) gets the
current hostname, (7) gets the new hostname from the hostname list and
(10-14) assigns the new hostname to the \ac{Raspi} where the script
will be executed.

\subsubsection{Updating Default Configuration Files and Scripts}
Besides the single script with its configuration file introduced up
to this point of our procedure, it is possible that the testbed
administrator may require to add other scripts to configure
his / her \ac{Raspi}s. We want
to ensure that all the \ac{Raspi} configuration scripts of any testbed
administrators are obtained in a simple way. We automate this task by
including the \texttt{\$\{ROOTDIR\}/home/pi/rasp\_config/update\_rasp\_config}
script in our procedure. The purpose of this script is to make all the
\ac{Raspi}s fetch all the configuration scripts located with the image
during a testbed start up.

In our case as the testbed administrator for presenting the procedure, we want
to fetch all our configuration scripts in
\texttt{\$\{URL\%/\}/raspi\_config.zip}. The update script
\textit{automatically} downloads all the required configuration files in
\texttt{rasp\_config.zip} file from a remote location. This is the same that
we manually did earlier to get our files but this will be made in an
automated way after booting up the system. This script content is:

\Suppressnumber\begin{lstlisting}[]
<@\textcolor{gray}{\$\{ROOTDIR\}/home/pi/rasp\_config/update\_rasp\_config}@>
<@\textcolor{gray}{
--------------------------------------------------------------------------------------------------}
\Reactivatenumber @>
#!/usr/bin/env bash

url="https://zenodo.org/record/154328/files/"
config_file="rasp_config.zip"

# Attempt to fetch new configuration files
if ! wget -q --show-progress -O /tmp/${config_file} ${url%/}/${config_file}; then
    echo "Warning: Unable to update rasp_config files"
    exit 1
fi

# Unzip and overwrite configurationn files to root's home directory
unzip -q -o /tmp/${config_file} -d /home/pi/
\end{lstlisting}
\FloatBarrier
\vspace{-5mm}

The update script lines (3-4) specifies the \ac{URL} and \texttt{.zip} file that
should be downloaded. Lines (7-10) downloads the configuration files to \texttt{/tmp}
folder in the corresponding \ac{Raspi}. It also prints a warning in case of
errors and line (13) unzips the files to \texttt{/home/pi/}, the \ac{Raspi} home
directory. Existing files and directories are simply overwritten.
%while ignoring existing files and / or directories there.

For the above scripts to work in the \ac{Raspi}s, it is required that the
\ac{Raspi}s \ac{MAC} addresses are found in \texttt{nodes.csv}. Also, it
should be noted that for other testbed administrators besides ourselves,
the \ac{URL} for file fetching and the configuration
scripts themselves can be modified to fit their requirements. If
required for a testbed administrator, the \texttt{rasp\_config.zip}
will need to be edited to include all the required configuration files
and scripts. Also, it might be needed to edit the \ac{URL} in the script
\texttt{update\_rasp\_config} to store and fetch from a different location.
Nevertheless, both the \ac{URL} and configuration files presented here can be
used as a starting boilerplate if desired.

\subsubsection{Configuration Scripts Execution Order}
To actually make the \ac{Raspi}s change hostnames and any other considered
configurations, we have to make each \ac{Raspi} call the above scripts when
it starts up. After finishing the setup process, all the unzipped files
presented in Section~\ref{sec:configuration_files_download} should be
locally available at each \ac{Raspi} after getting the root filesystem.
We first need to run the update script before running
any other configuration scripts. To do this after boot up, we include
a call for the update script in \texttt{\$\{ROOTDIR\}/etc/rc.local} before
\texttt{exit 0} in the file:

%before \texttt{exit 0} using a file editor. To do this, it is required
%to edit this file with root permissions. The resulting file should look like the following:

\begin{lstlisting}[]
# sed -i '/^exit 0/i bash /home/pi/rasp_config/update_rasp_config' ${ROOTDIR}/etc/rc.local
\end{lstlisting}
\FloatBarrier
\vspace{-5mm}

%\Suppressnumber\begin{lstlisting}[]
%<@\textcolor{gray}{\$\{ROOTDIR\}/etc/rc.local}@>
%<@\textcolor{gray}{
%---------------------------------------------}
%\Reactivatenumber @>
%...
%bash /home/pi/rasp_config/update_rasp_config
%...
%exit 0
%\end{lstlisting}
%\FloatBarrier
%\vspace{-5mm}

If it is required to have more configuration scripts, adding them
in the \texttt{rc.local} file difficults its maintenance by the testbed
administrator since this needs to be both in the image and the
downloaded \texttt{rasp\_config.zip}. To avoid this problem, we include
the \texttt{\$\{ROOTDIR\}/home/pi/rasp\_config/main} script which
calls all other configuration scripts (besides \texttt{update\_rasp\_config})
in a sequential order. This script content is:

\Suppressnumber\begin{lstlisting}[]
<@\textcolor{gray}{\$\{ROOTDIR\}/home/pi/rasp\_config/main}@>
<@\textcolor{gray}{
---------------------------------------------------------------------}
\Reactivatenumber @>
#!/usr/bin/env bash

bash /home/pi/rasp_config/set_hostname
# Any other required configuration scripts...
\end{lstlisting}
\FloatBarrier
\vspace{-5mm}

In this way, the automation process is simplified since we do not need to
modify \texttt{\$\{ROOTDIR\}/etc/rc.local} again after the image has been
written to the memory cards.
%In this way, the automation process is simplified since we only need to modify
%the image \texttt{\$\{ROOTDIR\}/etc/rc.local} file once to execute this script.
Now, we insert a call to the \texttt{main} script in \texttt{\$\{ROOTDIR\}/etc/rc.local} as follows:

%\todot{insert sed command below instead}
\begin{lstlisting}[]
# sed -i '/^exit 0/i bash /home/pi/rasp_config/main' ${ROOTDIR}/etc/rc.local
\end{lstlisting}
\FloatBarrier
\vspace{-5mm}

Finally, \texttt{\$\{ROOTDIR\}/etc/rc.local} should look like the following:

\Suppressnumber\begin{lstlisting}[]
<@\textcolor{gray}{\$\{ROOTDIR\}/etc/rc.local}@>
<@\textcolor{gray}{
---------------------------------------------}
\Reactivatenumber @>
...
bash /home/pi/rasp_config/update_rasp_config
bash /home/pi/rasp_config/main
exit 0
\end{lstlisting}
\FloatBarrier
\vspace{-5mm}

Notice that \texttt{set\_hostname} is now called by the \texttt{main}
script instead. The update script is still called directly. This ensures that
all configuration scripts are updated before executed. Changes to the update
script itself will first take effect at next system startup.

%\url{https://wiki.debian.org/QemuUserEmulation}
%\url{https://wiki.archlinux.org/index.php/Raspberry_Pi}
\subsection{Image Packages Updating by Changing the Apparent Root Directory}
Besides adding and configuring files within the image, the testbed
administrator may want to install and update the software packages
within the image before it is written to all the memory cards that goes
into the \ac{Raspi}s. From any Linux x86 machine as the testbed administrator
\ac{PC}, this can be done using \texttt{chroot} command in the
QEMU~\cite{QemuUserEmulation} hypervisor for \ac{ARM} processors.

\texttt{chroot} is a method in Linux that modifies the apparent root
filesystem location from \texttt{/} to any other path. Consequently, in
our case, we can use the Raspbian Lite image root filesystem within the
testbed administrator Linux distribution. Then, QEMU allows the execution
of commands for the \ac{Raspi} image (\ac{ARM} instructions) through the ones
from the testbed administrator \ac{PC} architecture. Due to the \ac{ARM}
processor that the \ac{Raspi}s employ, it is required to install the QEMU related software first and verify that QEMU is \ac{ARM} enabled. To do so,
run the following commands:

% CHROOT to OS image
\begin{lstlisting}[]
# apt-get install binfmt-support qemu qemu-user-static
# update-binfmts --display qemu-arm
qemu-arm (enabled):
     package = qemu-user-static
        type = magic
      offset = 0
       magic = \x7fELF\x01\x01\x01\x00\x00\x00\x00\x00\x00\x00\x00\x00\x02\x00\x28\x00
        mask = \xff\xff\xff\xff\xff\xff\xff\x00\xff\xff\xff\xff\xff\xff\xff\xff\xfe\xff\xff\xff
 interpreter = /usr/bin/qemu-arm-static
    detector =
\end{lstlisting}
\FloatBarrier
\vspace{-5mm}
% This was neccessary on arch
%# update-binfmts --importdir /var/lib/binfmts/ --import

In the previous output, the testbed administrator must be sure that the second
command writes \texttt{qemu-arm (enabled)} as indicated. If that is not the
case, then it should be possible to enable it by running:

\begin{lstlisting}[]
# update-binfmts --enable qemu-arm
\end{lstlisting}
\FloatBarrier
\vspace{-5mm}

Provided that qemu-arm is enabled, we should now be able to \texttt{chroot}
into our Raspbian lite image. There are a few commands to be
performed before actually changing root into the root partion of the image.
First, to get internet access from within the Raspbian lite image it is needed
to copy the testbed administrator Linux distribution \texttt{resolv.conf} file
into the image root filesystem. To do this, it is required to run the
following:

\begin{lstlisting}[]
$ cd $ROOTDIR
# cp /etc/resolv.conf ${ROOTDIR}/etc/resolv.conf
\end{lstlisting}
\FloatBarrier
\vspace{-5mm}

Now, because of the \ac{ARM} architecture, the
\texttt{/usr/bin/qemu-arm-static} command needs to be copied into the
image before continuing by running:

% CHROOT to OS image
\begin{lstlisting}[]
# cp /usr/bin/qemu-arm-static ${ROOTDIR}/usr/bin
\end{lstlisting}
\FloatBarrier
\vspace{-5mm}

Before changing root it is necessary to populate the directories
\texttt{proc}, \texttt{sys} and \texttt{dev} for the image to get control
as the testbed administrator apparent root filesystem. This is made by the
following commands:

% CHROOT to OS image
\begin{lstlisting}[]
# mount  -t proc proc proc/
# mount --bind /sys sys/
# mount --bind /dev dev/
# mount --bind /dev/pts dev/pts
\end{lstlisting}
\FloatBarrier
\vspace{-5mm}

%Finally, to change root the following needs to be run:
Finally, run the following command to change root:

\begin{lstlisting}[]
# chroot ${ROOTDIR} /usr/bin/qemu-arm-static /bin/bash
\end{lstlisting}
\FloatBarrier
\vspace{-5mm}

If successfully executed, our terminal should have changed the prompt
indicating that we are the root user in the Raspbian lite root filesystem as
the apparent root. In case the \texttt{chroot} command is not successful,
we provide an alternative command in Section~\ref{sec:chroot} of the Appendices.
To be aware of the mode that we are working now, we change
the prompt title to indicate that it is a \texttt{chroot} environment as
follows:

% Optionally, we may create a unique prompt to indicate we have changed root
\begin{lstlisting}[]
# export PS1="(chroot) $PS1"
\end{lstlisting}
\FloatBarrier
\vspace{-5mm}

The Raspbian lite image should now be possible to use almost as if it had
been booted in a \ac{Raspi}. A major difference is that the testbed
administrator \ac{PC} is likely significantly faster than a \ac{Raspi}.
Hence, enabling updates, upgrades and installing new software packages
should be faster than in a \ac{Raspi}. Still updating and upgrading
the packages for the \ac{Raspi} might take some amount of time.
To get the upgraded system packages, it is necessary to run:
%To update the system package list, run the following command:

% Update system
\begin{lstlisting}[]
(chroot) # apt-get update
(chroot) # apt-get upgrade
\end{lstlisting}
\FloatBarrier
\vspace{-5mm}

We further install some packages that we consider useful:
%Lets install some useful applications:
% Install packages
\begin{lstlisting}[]
(chroot) # apt-get install vim git screen
\end{lstlisting}
\FloatBarrier
\vspace{-5mm}

\texttt{vim} is the improved \texttt{vi} editor for Linux, \texttt{git}
for managing Git repositories and \texttt{screen}~\cite{gnu_screen} for better handling of
long-runnning processes. When writing the image to a memory card, all the
changes that have been made to the image so far, will exist in all
\ac{Raspi}s after fetching it.
