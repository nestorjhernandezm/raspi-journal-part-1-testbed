\label{sec:image_setup}
In this Section, we review the steps to create a common \ac{OS} image for
all the \ac{Raspi}s. The image setup is composed of three major steps:
Select and download the \ac{OS} image file, alter the image structure and
configure the \ac{OS} files. We proceed to detail all these steps providing
brief discussions to our setup choices when required. To perform these
steps, we indicate with command-line blocks the required sequential
commands to be typed by the testbed administrator in his / her \ac{PC}
to obtain the desired setting. In all the following sections, the number
sign (\#) and dollar sign (\$) will be used in the command blocks in the
paper to indicate if an operation needs to be run with root permissions or
common user permissions, respectively. The signs will be posted as the
terminal prompt in the command blocks.

\subsection{OS Selection and Download}

To get started, we first need to install an \ac{OS} that works properly
on all the \ac{Raspi} models. We will download and setup the image in
the testbed administrator \ac{PC} using a Debian-based distribution. We
use the popular Debian-based Raspbian Linux~\cite{raspbian} given that is
the recommended and default \ac{OS} for the \ac{Raspi}. Raspbian is made
available in two bundles: Raspbian and Raspbian lite. The difference
between the two, is that Raspbian contains a pre-installed desktop environment
for user interaction and Raspbian lite by default only permits to interact
through a command shell. Given that the \ac{Raspi}s in our testbed are not
connected to monitors, we decide to work with Raspbian lite. Still, if later
desired, a desktop environment can be installed using the package manager
as it will be shown in the configuration section.

% The overall procedure to customize the official Raspbian lite image
% are:
% \begin{enumerate}
%     \item Download Raspbian lite
%     \item Alter Raspbian lite. e.g. browse, modify, add, and delete files
%         in the official Raspbian lite image
%     \item Change root, i.e. change root filesystem into the Raspbian lite
%         image to update and install software packages
%     \item Write image to memory cards
% \end{enumerate}

%\subsection{Download Raspbian lite}
%To download the latest Raspbian lite, we go to the url
%\url{http://downloads.raspberrypi.org/raspbian_lite/images/}.

The latest Raspbian lite bundle can be downloaded from the Raspbian
official webpage~\cite{raspbian}. At the time of this writing, the latest
available bundle was \texttt{2016-05-27-raspbian-jessie-lite.zip}.
To ensure that the content of the bundle does not change, this procedure
is based on that particular version of Raspbian lite which we have
made available at~\cite{tunescode_webpage}. All other files used in this paper
are available there unless specified otherwise. To get started, the testbed
administrator must open a Linux shell (terminal) in and declare the environment
variables shown in the command block below. We show the procedure by
performing the role of the testbed administrator. To copy the commands from
this procedure to the testbed administrator terminal,
we strongly recommend to read this document with Adobe Acrobat Reader in Linux.
Otherwise, copied characters might not render properly.

%After the terminal has been opened, we start by declaring a few variables to reduce
%repeated typing. The first variables we declare is the Raspbian image name.
%Notice that the extension has been omitted. This is because the image has been
%packed into a zip file. The other variable we declare is a working directory.
%This is where we will download the image to and work on it. In other words, it
%will be stored in \texttt{/home/<username>/Raspbian}

%To download Raspbian lite, we go to the \url{http://downloads.raspberrypi.org}.
%There should be a folder called raspbian\lite and
%to find the download page at which Raspbian lite should be located. Instead of
%downloading with the browser, we just extract the download url to the latest
%image. At the time of this writing, that was

% Define variable]
\begin{lstlisting}[]
$ export URL="http://kom.aau.dk/project/TuneSCode/raspi/"
$ export IMAGE="2016-05-27-raspbian-jessie-lite"
$ export WORKDIR="${HOME}/Raspbian"
\end{lstlisting}
\FloatBarrier
\vspace{-5mm}

In this code block, the \texttt{URL} and \texttt{IMAGE} variables specify where
the Linux bundle is located and \texttt{WORKDIR} specifies a working
directory where the Raspbian bundle will be downloaded and customized.
Notice that even though we use the \$ and \# signs in the shell, in general
these signs will be particular to the testbed administrator \ac{OS} shell.
Next, we create the working directory and go to it with the \texttt{cd}
command. To download the image, we utilize the \texttt{wget} command before
unpacking the \texttt{zip} file as follows:

% Running a command as root can on most systems be done by putting
% \texttt{sudo} in front of the command. This is illustrated in the
% following code block
% with the command \texttt{whoami} that prints the username. Lines within a code
% block without leading \$ or \# is terminal output or content of a file.

% Root example
% \begin{lstlisting}[]
% $ whoami
% <USERNAME>
% $ sudo whoami
% root
% \end{lstlisting}
% \FloatBarrier
% \vspace{-5mm}

% Download and unpack image
\begin{lstlisting}[]
$ mkdir -p ${WORKDIR}
$ cd ${WORKDIR}
$ wget ${URL%/*}/${IMAGE}.zip
$ unzip ${IMAGE}.zip
\end{lstlisting}
\FloatBarrier
\vspace{-5mm}

%A more recent version of Raspbian lite may be available at~\cite{raspbian}.
%\url{http://downloads.raspberrypi.org/raspbian_lite/images}.

\subsection{Image Customization}

After Raspbian lite has been unpacked, there should be an \texttt{.img}
file in the \texttt{WORKDIR} directory. Here, \texttt{fdisk} can be used to
display the content of the image. To do this and obtain administrative
information, we parse the arguments \texttt{-u sectors} to display the
sizes in sectors and \texttt{-l} to display the partitions within the
image. The \texttt{fdisk} command should output to the terminal something
similar to:

% Check out the image
% The dollar hack was to fix vim syntax
\begin{lstlisting}[literate={DOLLAR}{\$}1]
DOLLAR fdisk -u sectors -l ${IMAGE}.img
Disk 2016-05-27-raspbian-jessie-lite.img: 1.3 GiB, 1387266048 bytes, 2709504 sectors
Units: sectors of 1 * 512 = 512 bytes
Sector size (logical/physical): 512 bytes / 512 bytes
I/O size (minimum/optimal): 512 bytes / 512 bytes
Disklabel type: dos
Disk identifier: 0x6fcf21f3

Device                               Boot  Start     End Sectors  Size Id Type
2016-05-27-raspbian-jessie-lite.img1        8192  137215  129024   63M  c W95 FAT32 (LBA)
2016-05-27-raspbian-jessie-lite.img2      137216 2709503 2572288  1.2G 83 Linux
\end{lstlisting}
\FloatBarrier

The output provides relevant information about the image. The image is in
total 2709504 sectors (1.3GiB) in size and contains two partitions. The
first partition starts at sector 8192 and the other partition starts at
sector 137216. The first partition type is FAT32 with a size of 63 MB
and the second partition is a Linux one with a size of 1.2 GB. This
indicates that the first partition is a boot partition and the second
one is a traditional Linux root filesystem, i.e. \texttt{/}.

\subsection{Image Resizing}
Given that we want to customize the files stored in the \ac{Raspi}s,
we need to resize the image file since 1.2GB might not be enough to store
the root filesystem due to the total size of the additional packages. Thus,
we need to increase the partition size. The following procedure illustrates
how the image and its root filesystem can be expanded by one \ac{GB}.
First, to expand the image one \ac{GB}, we execute:

\begin{lstlisting}[]
$ dd if=/dev/zero bs=1M count=1024 >> ${IMAGE}.img && sync
\end{lstlisting}
\FloatBarrier
\vspace{-5mm}

Later, we use \texttt{fdisk} with the same arguments as before to see that
the image is now one \ac{GB} larger:
\begin{lstlisting}[literate={DOLLAR}{\$}1]
DOLLAR fdisk -u sectors -l ${IMAGE}.img
Disk 2016-05-27-raspbian-jessie-lite.img: 2.3 GiB, 2461007872 bytes, 4806656 sectors
Units: sectors of 1 * 512 = 512 bytes
Sector size (logical/physical): 512 bytes / 512 bytes
I/O size (minimum/optimal): 512 bytes / 512 bytes
Disklabel type: dos
Disk identifier: 0x6fcf21f3

Device                               Boot  Start     End Sectors  Size Id Type
2016-05-27-raspbian-jessie-lite.img1        8192  137215  129024   63M  c W95 FAT32 (LBA)
2016-05-27-raspbian-jessie-lite.img2      137216 2709503 2572288  1.2G 83 Linux
\end{lstlisting}
\FloatBarrier
\vspace{-5mm}

Now, in this command block output, we observe that the change has taken
effect by noticing the total available image size is 2.3GiB. To expand the
root filesystem, it is required to first remove the Linux partition
and then add it again with one \ac{GB} more. To do this, we make use
again of \texttt{fdisk} in command mode to alter the partition table.
For this, we pass the arguments of the command mode of \texttt{fdisk}
through the \texttt{echo} command in Linux and the \texttt{|} operator
(pipe) as follows:

\begin{lstlisting}[]
$ (echo d; echo 2; echo n; echo ; echo ; echo 137216; echo ; echo w) | fdisk ${IMAGE}.img
\end{lstlisting}
\FloatBarrier
\vspace{-5mm}

The \texttt{echo} commands within the parenthesis are interpreted as
key-presses in the \texttt{fdisk} commmand mode. They (i) delete partition
number 2, (ii) create a new primary type partition by default, (iii) set the
new partition starting point (137216) and (iv) write new partition table to
the image file. The previously shown command options of \texttt{fdisk}
should work for both old and new versions of the commands. However, if this
is not the case, the testbed administrator can do these commands manually.
In this case, it is only required to delete the old root
filesystem partition and create the new one starting from the same sector
number. In our case, if the partitions commands were correct, the output
should be following:

\begin{lstlisting}[]

Command (m for help): Partition number (1-4):
Command (m for help): Partition type:
   p   primary (1 primary, 0 extended, 3 free)
   e   extended
Select (default p): Using default response p
Partition number (1-4, default 2): Using default value 2
First sector (2048-4806655, default 2048): Last sector, +sectors or
+size{K,M,G} (137216-4806655, default 4806655): Using default value 4806655

Command (m for help): The partition table has been altered!

Syncing disks.

\end{lstlisting}
\FloatBarrier
\vspace{-5mm}

\subsection{Loopback Device Setup}
After successfully resizing the image file, we use a loopback device to make
the Raspbian image available as a block device in the filesystem. For this
command to work, the tested administrator distribution must have the
\texttt{util-linux} package with version 2.21 or higher. Otherwise, the
\texttt{-P} argument of \texttt{losetup} will appear as invalid.


\begin{lstlisting}[]
# export DEV=$(sudo losetup --show -f -P ${IMAGE}.img); echo $DEV
/dev/loop0
\end{lstlisting}
\FloatBarrier
\vspace{-5mm}

If the previous command was succesful, the \texttt{lsblk} command can be used
to list the available block devices in the filesystem as follows:
%Use \texttt{lsblk} to view the partitions:

\begin{lstlisting}[]
# lsblk
NAME      MAJ:MIN RM  SIZE RO TYPE MOUNTPOINT
...
loop0       7:0    0  2.3G  0 loop
|-loop0p1 259:2    0   63M  0 loop
|-loop0p2 259:3    0  2.2G  0 loop
...
\end{lstlisting}
\FloatBarrier
\vspace{-5mm}

The image block device appears as \texttt{/dev/loop0}. This block device has
two partitions associated to it, e.g. \texttt{loop0p1} and \texttt{loop0p2}.
Finally, we check the block device status with \texttt{e2fsck} and
\texttt{resize2fs}, respectively in the command block:

\begin{lstlisting}[]
# e2fsck -f ${DEV}p2
e2fsck 1.42.8 (20-Jun-2013)
Pass 1: Checking inodes, blocks, and sizes
Pass 2: Checking directory structure
Pass 3: Checking directory connectivity
Pass 4: Checking reference counts
Pass 5: Checking group summary information
/dev/loop0p2: 35392/80480 files (0.1% non-contiguous), 201968/321536 blocks
# resize2fs ${DEV}p2
resize2fs 1.42.8 (20-Jun-2013)
Resizing the filesystem on /dev/loop0 to 583680 (4k) blocks.
The filesystem on /dev/loop0 is now 583680 blocks long.
\end{lstlisting}
\FloatBarrier

\subsection{Block Devices Mounting}
For browsing and altering the files in the image, we mount the block
device partitions into a particular path of our \texttt{WORKDIR} in order
to customize them. We proceed first to mount the block device partition that
contains the root filesystem and later the boot partition. This is done by
creating an empty directory that it is used as a mountpoint. We name it
\texttt{root} and create it in the working directory before mounting the
root filesystem onto the mountpoint. Thus, we mount the root filesystem
as follows:

\begin{lstlisting}[]
$ export ROOTDIR="${WORKDIR}/root"
$ mkdir -p ${ROOTDIR}
# mount ${DEV}p2 ${ROOTDIR}
\end{lstlisting}
\FloatBarrier
\vspace{-5mm}
%# mount -o loop,offset=$((137216*512)) ${IMAGE}.img ${ROOTDIR}

The root filesystem mounted in \texttt{ROOTDIR}, has already a boot
directory that can be used as the mount point for the boot partition
in the block device \texttt{loop0p1}. This is convenient because
the final edited partition from \texttt{ROOTDIR/boot} will be mounted on
this same directory when a \ac{Raspi} starts up with a memory card
containing the raw final image. Hence, to mount boot partition we do:
%It is therefore the natural place to mount it for later purposes.

\begin{lstlisting}[]
# mount ${DEV}p1 ${ROOTDIR}/boot
\end{lstlisting}
\FloatBarrier
\vspace{-5mm}
%# mount -o loop,offset=$((8192*512)) ${IMAGE}.img ${ROOTDIR}/boot
%We can now change all the files in the disk image as desired.

In this way, it is now possible to change all files within the Raspbian
image as desired by editing the files in \texttt{ROOTDIR}. We take
advantage of this to edit configuration files, append new files and even
update and install packages.

\subsection{Image OS Files and Configuration Scripts Setup}
In general, the \ac{Raspi}s should be setup as similar as possible. However,
some particularities exist to differentiate the devices in principle. Also,
scripts containing further configurations for the \ac{Raspi}s are desirable
to be distributed as part of the common image. Therefore, here we present
the steps to setup basic properties of the \ac{Raspi}s and distributing
configuration scripts to each of them through the image. For this, we first
indicate how to obtain and put our configuration scripts in the image. Later,
we describe the tasks performed by these configuration scripts. Finally, we
indicate how and in which order are the scripts executed to configure all the
devices. Nevertheless, any testbed administrator might modify or include other
tasks besides the presented in this procedure according to his / her needs
as we will show.

\subsubsection{Image Default Configuration Scripts Download}
\label{sec:configuration_files_download}
In our case, we have our default configuration scripts stored in a
\texttt{rasp\_config.zip} file located in the same \ac{URL} where the image
was retrieved from, i.e. the one in the environment variable \texttt{URL}.
We first manually download this compressed file with \texttt{wget} and
extract it locally into our Raspbian lite image. These commands and the
output of the last one are shown as follows:

% Get the configuration files that are compressed in a remote server
\begin{lstlisting}[]
$ wget ${URL%/*}/rasp_config.zip
$ unzip rasp_config.zip -d ${ROOTDIR}/home/pi/
Archive:  rasp_config.zip
   creating: ${ROOTDIR}/home/pi/rasp_config/
  inflating: ${ROOTDIR}/home/pi/rasp_config/nodes.csv
  inflating: ${ROOTDIR}/home/pi/rasp_config/set_hostname
  inflating: ${ROOTDIR}/home/pi/rasp_config/main
  inflating: ${ROOTDIR}/home/pi/rasp_config/update_rasp_config
\end{lstlisting}
\FloatBarrier
\vspace{-5mm}

The unzippped files are one configuration file and three configuration
scripts put in the newly created \texttt{\${ROOTDIR}/home/pi/rasp\_config/}
folder in the image. In what follows, we describe which features do we
require all the \ac{Raspi}s to have and how are they achieved with
these configuration scripts.

\subsubsection{Device Hostnames}
Among the settings that we want to be different in the devices is
their hostname. The hostname helps the user to physically distinguish the
devices from each other. Thus, in our case we require the devices in our
testbed to have different hostnames. Hence, we define the hostnames based
on the \ac{MAC} addresses of the \ac{Raspi}s wired Ethernet interface.

Prior to this stage, the \ac{MAC} address of a network card can be found
using the command \texttt{ifconfig} or \texttt{ip addr} on a given
\ac{Raspi}. We store the \ac{MAC} addresses and hostnames of the
\ac{Raspi}s in the configuration file
\texttt{\${ROOTDIR}/home/pi/rasp\_config/nodes.csv}. A sample of our file
is shown as follows:

% MAC and Hostname file
\Suppressnumber\begin{lstlisting}[]
<@\textcolor{gray}{\$\{ROOTDIR\}/home/pi/rasp\_config/nodes.csv}@>
<@\textcolor{gray}{
--------------------------------------------------------------------------------}
\Reactivatenumber @>
# Ethernet MAC    Hostname
b8:27:eb:5b:da:20 rasp00
b8:27:eb:7b:c3:91 rasp01
b8:27:eb:54:9c:64 rasp02
b8:27:eb:95:bd:11 rasp03
\end{lstlisting}
\FloatBarrier
\vspace{-5mm}

The testbed administrator has to insert the \ac{MAC} addresses and hostnames
of his/her \ac{Raspi}s obtained previously in the format shown in the
configuration file. In the file format, on each line there is a \ac{MAC}
address and the corresponding hostname for the given \ac{Raspi}. For
simplicity in the example above, we just include three \ac{Raspi}s.
This list file will be employed by the
\texttt{\${ROOTDIR}/home/pi/rasp\_config/set\_hostname}
\ac{Bash} script to assign the hostname of each \ac{Raspi}.
The script content is the following:

% Set hostname
\Suppressnumber\begin{lstlisting}[]
<@\textcolor{gray}{\$\{ROOTDIR\}/home/pi/rasp\_config/set\_hostname}@>
<@\textcolor{gray}{
-------------------------------------------------------------------------------------}
\Reactivatenumber @>
#!/usr/bin/env bash

script_path="$(dirname $(realpath $0))"
config_file=${script_path}/nodes.csv
mac=$(cat /sys/class/net/eth0/address)
old_hostname=$(hostname)
new_hostname=$(grep $mac $config_file | cut -f2 -d' ')

# Assign hostname found in nodes.csv
if [ ! -z ${new_hostname} ]; then
    echo ${new_hostname} > /etc/hostname
    hostname ${new_hostname}
    sed -i.old -e "s:${old_hostname}:${new_hostname}:g" /etc/hosts
fi
\end{lstlisting}
\FloatBarrier
\vspace{-5mm}

The script (in lines): (1) tells the system to interpret the script
using \ac{Bash}, (3-4) gets the path to the script itself and the list of
hostnames, (5) gets the \ac{MAC} address of the node itself, (6) gets the
current hostname, (7) gets the new hostname from the hostname list and
(10-14) assigns the new hostname to the \ac{Raspi} where the script
will be executed.

\subsubsection{Updating Default Configuration Files and Scripts}
Besides the single script with its configuration file introduced up
to this point of our procedure, it is possible that the testbed
administrator may require to add other scripts to configure
his / her \ac{Raspi}s. Hence, we want
to ensure that all the \ac{Raspi} configuration scripts of any testbed
administrators are obtained in a simple way. We automate this task by
including the \texttt{\$\{ROOTDIR\}/home/pi/rasp\_config/update\_rasp\_config}
script in our procedure. The purpose of this script is to make all the
\ac{Raspi}s fetch all the configuration scripts located with the image
during a testbed start up.

In our case as the testbed administrator for presenting the procedure, we want
to fetch all our configuration scripts in
\texttt{\$\{URL\%/*\}/raspi\_config.zip}. Thus, the update script
\textit{automatically} downloads all the required configuration files in
\texttt{rasp\_config.zip} file from a remote location. This is the same that
we manually did earlier to get our files but this will be made in an
automated wat after booting up the system. This script contents are:

\Suppressnumber\begin{lstlisting}[]
<@\textcolor{gray}{\$\{ROOTDIR\}/home/pi/rasp\_config/update\_rasp\_config}@>
<@\textcolor{gray}{
--------------------------------------------------------------------------------------------------}
\Reactivatenumber @>
#!/usr/bin/env bash

url="http://kom.aau.dk/project/TuneSCode/raspi/"
config_file="rasp_config.zip"

# Attempt to fetch new configuration files
if ! wget -q --show-progress -O /tmp/${config_file} ${url%/}/${config_file}; then
    echo "Warning: Unable to update rasp_config files"
    exit 1
fi

# Unzip and overwrite configurationn files to root's home directory
unzip -q -o /tmp/${config_file} -d /home/pi/
\end{lstlisting}
\FloatBarrier
\vspace{-5mm}

The update script: (3-4) checks for the \ac{URL} and \texttt{.zip} file that
should be downloaded, (7-10) downloads the configuration files to \texttt{/tmp}
folder in the corresponding \ac{Raspi}. It also prints a warning in case of
errors and (13) unzips the files to \texttt{/home/pi/}, the \ac{Raspi} home
directory, while ignoring existing files and / or directories there.

For the above scripts to work in the \ac{Raspi}s, it is required that the
\ac{Raspi}s \ac{MAC} addresses are found in \texttt{nodes.csv}. Also, it
should be noted that for other testbed administrators besides ourselves,
the \ac{URL} for file fetching and the configuration
scripts themselves can be modified to fit their requirements. Thus, if
required for a testbed administrator, the \texttt{rasp\_config.zip}
will need to be edited to include all the required configuration files
and scripts. Also, it might be needed to edit the \ac{URL} in the script
\texttt{update\_rasp\_config} to store and fetch from a different location.
Nevertheless, both the \ac{URL} and configuration files presented here can be
used as a starting boilerplate if desired.

\subsubsection{Configuration Scripts Execution Order}
To actually make the \ac{Raspi}s change hostnames and any other considered
configurations, we have to make each \ac{Raspi} call the above scripts when
it starts up. After finishing the setup process, all the unzipped files
presented in Section~\ref{sec:configuration_files_download} should be
locally available at each \ac{Raspi} after getting the root filesystem.
Therefore, we first need to run the update script before running
any other configuration scripts. To do this after boot up, we include
a call for the update script in the image \texttt{\$\{ROOTDIR\}/etc/rc.local}
file before \texttt{exit 0} using a file editor. To do this, it is required
to edit this file as root. The resulting file should look like the following:

\Suppressnumber\begin{lstlisting}[]
<@\textcolor{gray}{\$\{ROOTDIR\}/etc/rc.local}@>
<@\textcolor{gray}{
---------------------------------------------}
\Reactivatenumber @>
...
bash /home/pi/rasp_config/update_rasp_config
...
exit 0
\end{lstlisting}
\FloatBarrier
\vspace{-5mm}

If it is required to have more configuration scripts, adding them
in the \texttt{rc.local} file difficults its maintenance by the testbed
administrator since this needs to be both in the image and the
downloaded \texttt{rasp\_config.zip}. To avoid this problem, we include
the \texttt{\$\{ROOTDIR\}/home/pi/rasp\_config/main} script which
calls all other configuration scripts (besides \texttt{update\_rasp\_config})
in sequential order. This script content is the following:

\Suppressnumber\begin{lstlisting}[]
<@\textcolor{gray}{\$\{ROOTDIR\}/home/pi/rasp\_config/main}@>
<@\textcolor{gray}{
---------------------------------------------------------------------}
\Reactivatenumber @>
#!/usr/bin/env bash

bash /home/pi/rasp_config/set_hostname
# Any other required configuration scripts...
\end{lstlisting}
\FloatBarrier
\vspace{-5mm}

In this way, the automation process is simplify since we only need to modify
the image \texttt{\$\{ROOTDIR\}/etc/rc.local} file once to execute this script:

\Suppressnumber\begin{lstlisting}[]
<@\textcolor{gray}{\$\{ROOTDIR\}/etc/rc.local}@>
<@\textcolor{gray}{
-----------------------------------------------------}
\Reactivatenumber @>
...
bash /home/pi/rasp_config/update_rasp_config
bash /home/pi/rasp_config/main
...
exit 0
\end{lstlisting}
\FloatBarrier
\vspace{-5mm}
Notice that \texttt{set\_hostname} is now called by the \texttt{main}
script instead. The update script is still called directly. This ensures that
all configuration scripts are updated before executed. Changes to the update
script itself will first take effect at next system startup.

\subsection{Image Package Updating}
Besides adding and configuring files within the image, the testbed
administrator can also install and update the software packages within
the image which will be illustrated in this subsection.

It may be desired to pre-install some programs in the image before it is
written to all the memory cards that goes into the \ac{Raspi}s.
This can be done from any linux x86 machine using QEMU Chroot~\cite{QemuUserEmulation}
(change root).
%\url{https://wiki.debian.org/QemuUserEmulation}
%\url{https://wiki.archlinux.org/index.php/Raspberry_Pi}

Due to the \ac{ARM} processor that \ac{Raspi}s are using, it is required to
install some additional software:
%Because \ac{Raspi}s are equipped with an \ac{ARM} processor
%%it is not as
%straightforward as performing the task for two Linux systems of the same
%architecture.

% CHROOT to OS image
\begin{lstlisting}[]
# apt-get install binfmt-support qemu qemu-user-static
# update-binfmts --display qemu-arm
qemu-arm (enabled):
     package = qemu-user-static
        type = magic
      offset = 0
       magic = \x7fELF\x01\x01\x01\x00\x00\x00\x00\x00\x00\x00\x00\x00\x02\x00\x28\x00
        mask = \xff\xff\xff\xff\xff\xff\xff\x00\xff\xff\xff\xff\xff\xff\xff\xff\xfe\xff\xff\xff
 interpreter = /usr/bin/qemu-arm-static
    detector =
\end{lstlisting}
\FloatBarrier
\vspace{-5mm}
% This was neccessary on arch
%# update-binfmts --importdir /var/lib/binfmts/ --import

Make sure the second command writes "enabled" as the output above. If that
is not the case, then try enabling it:

\begin{lstlisting}[]
# update-binfmts --enable qemu-arm
\end{lstlisting}
\FloatBarrier
\vspace{-5mm}

Provided that qemu-arm is enabled, we should now be able to change root (chroot)
into our Raspbian lite image. Change root is a method in Linux that enables
the root (/) to be changed. Thus, enabling a Linux installation within a
Linux installation.

There are a few commands to be performed before actually changing root into the
root partion of the image.
First, to get internet access from within the Raspbian lite image it is needed
to copy our \texttt{resolv.conf} into the filesystem. Line 1 change directory
into root filesystem of the image. The second line copies \texttt{resolv.conf}
from the testbed administrators \ac{PC} to the image.

% CHROOT to OS image
\begin{lstlisting}[]
$ cd $ROOTDIR
# cp /etc/resolv.conf ${ROOTDIR}/etc/resolv.conf
\end{lstlisting}
\FloatBarrier
\vspace{-5mm}

Now, because of the \ac{ARM} architecture,
\texttt{/usr/bin/qemu-arm-static} needs to be copied into the image before continuing:

% CHROOT to OS image
\begin{lstlisting}[]
# cp /usr/bin/qemu-arm-static ${ROOTDIR}/usr/bin
\end{lstlisting}
\FloatBarrier
\vspace{-5mm}

The final preparation before changing root is to populate the directories \texttt{proc},
\texttt{sys}, and \texttt{dev}:

% CHROOT to OS image
\begin{lstlisting}[]
# mount  -t proc proc proc/
# mount --bind /sys sys/
# mount --bind /dev dev/
# mount --bind /dev/pts dev/pts
\end{lstlisting}
\FloatBarrier
\vspace{-5mm}

Finally, it is time to change root.
%Finally, the filesystem is ready for us to change root.
This can be done with \texttt{proot}. It may be required to install
\texttt{proot} using \texttt{apt-get install proot}:

% CHROOT to OS image
\begin{lstlisting}[]
# proot -q qemu-arm-static -S ${ROOTDIR}
\end{lstlisting}
\FloatBarrier
\vspace{-5mm}

Most available material online uses the more known \texttt{chroot} command as
written in the code block below instead of \texttt{proot}. This did not work
correctly in our machines, but we present the command as an alternative to
\texttt{proot} in case the testbed administrator should be in the opposite
situration where \texttt{proot} is not working.

% CHROOT to OS image
\begin{lstlisting}[]
# chroot ${ROOTDIR} /usr/bin/qemu-arm-static /bin/bash    (ALTERNATIVE TO proot)
\end{lstlisting}
\FloatBarrier
\vspace{-5mm}

If everything went well, we should now be \texttt{chrooted} into the Raspbian lite
filesystem under its root. Optionally, change the prompt title to indicate that it
is a \texttt{chrooted} environment:

% Optionally, we may create a unique prompt to indicate we have changed root
\begin{lstlisting}[]
# export PS1="(chroot) $PS1"
\end{lstlisting}
\FloatBarrier
\vspace{-5mm}

The Raspbian lite system image should now be possible to use almost as if it
had been booted in a \ac{Raspi}.
A major difference is that the testbed administrator's \ac{PC} is likely
significantly faster than a \ac{Raspi}.
%The only difference is that the laptop that we are using is
%significantly faster than a \ac{Raspi}.
Thus enabling updates, upgrades, and installing new software packages much
faster than in the \ac{Raspi}s.
%Thus enabling us to for example update,
%upgrade, and install software packages:

% Update system
\begin{lstlisting}[]
(chroot) # apt-get update
(chroot) # apt-get upgrade
\end{lstlisting}
\FloatBarrier
\vspace{-5mm}

We also install a few packages that we find useful:
%Lets install some useful applications:
% Install packages
\begin{lstlisting}[]
(chroot) # apt-get install vim git screen
\end{lstlisting}
\FloatBarrier
\vspace{-5mm}

When writing the image to a memory card, all these change that has been made
to the image so far will exists in all \ac{Raspi}s.
%All the changes that are made here will exists in all \ac{Raspi}s when the
%image is written to a memory card.