\label{sec:image_setup}
In this Section, we review the steps to create a common \ac{OS} image for
all the \ac{Raspi}s. The image setup is composed of three major steps:
Select and download the \ac{OS} image file, alter the image structure and
configure the \ac{OS} files. We proceed to detail all these steps providing
brief discussions to our setup choices when required. To perform these
steps, we indicate with command-line blocks the required sequential
commands to be typed by the testbed administrator in its \ac{PC} to obtain
the desired setting. In all the following sections, the number sign
(\#) and dollar sign (\$) will be used in the command blocks in the paper
to indicate if an operation needs to be run with root permissions or
common user permissions, respectively. The signs will be posted as the
terminal prompt in the command blocks.

\subsection{OS Selection and Download}

To get started, we first need to install an \ac{OS} that works properly
on all the \ac{Raspi} models. We will download and setup the image in
the testbed administrator \ac{PC} using a Debian-based distribution. We
use the popular Debian-based Raspbian Linux~\cite{raspbian} given that is
the recommended and default \ac{OS} for the \ac{Raspi}. Raspbian is made
available in two bundles: Raspbian and Raspbian lite. The difference
between the two, is that Raspbian contains a pre-installed desktop environment
for user interaction and Raspbian lite by default only permits to interact
through a command shell. Given that the \ac{Raspi}s in our testbed are not
connected to monitors, we decide to work with Raspbian lite. Still, if later
desired, a desktop environment can be installed using the package manager
as it will be shown in the configuration section.

% The overall procedure to customize the official Raspbian lite image
% are:
% \begin{enumerate}
%     \item Download Raspbian lite
%     \item Alter Raspbian lite. e.g. browse, modify, add, and delete files
%         in the official Raspbian lite image
%     \item Change root, i.e. change root filesystem into the Raspbian lite
%         image to update and install software packages
%     \item Write image to memory cards
% \end{enumerate}

%\subsection{Download Raspbian lite}
%To download the latest Raspbian lite, we go to the url
%\url{http://downloads.raspberrypi.org/raspbian_lite/images/}.

The latest Raspbian lite bundle can be downloaded from the Raspbian
official webpage~\cite{raspbian}. At the time of this writing, the latest
available bundle was \texttt{2016-05-27-raspbian-jessie-lite.zip}.
To ensure that the content of the bundle does not change, this procedure
is based on that particular version of Raspbian lite which we have
made available at~\cite{tunescode_webpage}. All other files used in this paper
are available there unless specified otherwise. To get started, the testbed
administrator must open a Linux shell (terminal) in and declare the environment
variables shown in the command block below. We show the procedure by
performing the role of the testbed administrator. To copy the commands from
this procedure to the testbed administrator terminal,
we strongly recommend to read this document with Adobe Acrobat Reader in Linux.
Otherwise, copied characters might not render properly.

%After the terminal has been opened, we start by declaring a few variables to reduce
%repeated typing. The first variables we declare is the Raspbian image name.
%Notice that the extension has been omitted. This is because the image has been
%packed into a zip file. The other variable we declare is a working directory.
%This is where we will download the image to and work on it. In other words, it
%will be stored in \texttt{/home/<username>/Raspbian}

%To download Raspbian lite, we go to the \url{http://downloads.raspberrypi.org}.
%There should be a folder called raspbian\lite and
%to find the download page at which Raspbian lite should be located. Instead of
%downloading with the browser, we just extract the download url to the latest
%image. At the time of this writing, that was

% Define variable]
\begin{lstlisting}[]
$ export URL="http://kom.aau.dk/project/TuneSCode/raspi/"
$ export IMAGE="2016-05-27-raspbian-jessie-lite"
$ export WORKDIR="${HOME}/Raspbian"
\end{lstlisting}
\FloatBarrier
\vspace{-5mm}

In this code block, the \texttt{URL} and \texttt{IMAGE} variables specify where
the Linux bundle is located and \texttt{WORKDIR} specifies a working
directory where the Raspbian bundle will be downloaded and customized.
Notice that even though we use the \$ and \# signs in the shell, in general
these signs will be particular to the testbed administrator \ac{OS} shell.
Next, we create the working directory and go to it with the \texttt{cd}
command. To download the image, we utilize the \texttt{wget} command before
unpacking the \texttt{zip} file as follows:

% Running a command as root can on most systems be done by putting
% \texttt{sudo} in front of the command. This is illustrated in the
% following code block
% with the command \texttt{whoami} that prints the username. Lines within a code
% block without leading \$ or \# is terminal output or content of a file.

% Root example
% \begin{lstlisting}[]
% $ whoami
% <USERNAME>
% $ sudo whoami
% root
% \end{lstlisting}
% \FloatBarrier
% \vspace{-5mm}

% Download and unpack image
\begin{lstlisting}[]
$ mkdir -p ${WORKDIR}
$ cd ${WORKDIR}
$ wget ${URL%/*}/${IMAGE}.zip
$ unzip ${IMAGE}.zip
\end{lstlisting}
\FloatBarrier
\vspace{-5mm}

%A more recent version of Raspbian lite may be available at~\cite{raspbian}.
%\url{http://downloads.raspberrypi.org/raspbian_lite/images}.

\subsection{Image Customization}

After Raspbian lite has been unpacked, there should be an \texttt{.img}
file in the \texttt{WORKDIR} directory. Here, \texttt{fdisk} can be used to
display the content of the image. To do this and obtain administrative
information, we parse the arguments \texttt{-u sectors} to display the
sizes in sectors and \texttt{-l} to display the partitions within the
image. The \texttt{fdisk} command should output to the terminal something
similar to:

% Check out the image
% The dollar hack was to fix vim syntax
\begin{lstlisting}[literate={DOLLAR}{\$}1]
DOLLAR fdisk -u sectors -l ${IMAGE}.img
Disk 2016-05-27-raspbian-jessie-lite.img: 1.3 GiB, 1387266048 bytes, 2709504 sectors
Units: sectors of 1 * 512 = 512 bytes
Sector size (logical/physical): 512 bytes / 512 bytes
I/O size (minimum/optimal): 512 bytes / 512 bytes
Disklabel type: dos
Disk identifier: 0x6fcf21f3

Device                               Boot  Start     End Sectors  Size Id Type
2016-05-27-raspbian-jessie-lite.img1        8192  137215  129024   63M  c W95 FAT32 (LBA)
2016-05-27-raspbian-jessie-lite.img2      137216 2709503 2572288  1.2G 83 Linux
\end{lstlisting}
\FloatBarrier

The output provides relevant information about the image. The image is in
total 2709504 sectors (1.3GiB) in size and contains two partitions. The
first partition starts at sector 8192 and the other partition starts at
sector 137216. The first partition type is FAT32 with a size of 63 MB
and the second partition is a Linux one with a size of 1.2 GB. This
indicates that the first partition is a boot partition and the second
one is a traditional Linux root filesystem, i.e. \texttt{/}.

\subsection{Image Resizing}
Given that we want to customize the files stored in the \ac{Raspi}s,
we need to resize the image file since 1.2GB might not be enough to store
the root filesystem due to the total size of the additional packages. Thus,
we need to increase the partition size. The following procedure illustrates
how the image and its root filesystem can be expanded by one \ac{GB}.
First, to expand the image one \ac{GB}, we execute:

\begin{lstlisting}[]
$ dd if=/dev/zero bs=1M count=1024 >> ${IMAGE}.img && sync
\end{lstlisting}
\FloatBarrier
\vspace{-5mm}

Later, we use \texttt{fdisk} with the same arguments as before to see that
the image is now one \ac{GB} larger:
\begin{lstlisting}[literate={DOLLAR}{\$}1]
DOLLAR fdisk -u sectors -l ${IMAGE}.img
Disk 2016-05-27-raspbian-jessie-lite.img: 2.3 GiB, 2461007872 bytes, 4806656 sectors
Units: sectors of 1 * 512 = 512 bytes
Sector size (logical/physical): 512 bytes / 512 bytes
I/O size (minimum/optimal): 512 bytes / 512 bytes
Disklabel type: dos
Disk identifier: 0x6fcf21f3

Device                               Boot  Start     End Sectors  Size Id Type
2016-05-27-raspbian-jessie-lite.img1        8192  137215  129024   63M  c W95 FAT32 (LBA)
2016-05-27-raspbian-jessie-lite.img2      137216 2709503 2572288  1.2G 83 Linux
\end{lstlisting}
\FloatBarrier
\vspace{-5mm}

Now, in this command block output, we observe that the change has taken
effect by noticing the total available image size is 2.3GiB. To expand the
root filesystem, it is required to first remove the Linux partition
and then add it again with one \ac{GB} more. To do this, we make use
again of \texttt{fdisk} in command mode to alter the partition table.
For this, we pass the arguments of the command mode of \texttt{fdisk}
through the \texttt{echo} command in Linux and the \texttt{|} operator
(pipe) as follows:

\begin{lstlisting}[]
$ (echo d; echo ; echo n; echo p; echo ; echo 137216; echo ; echo p; echo w) | fdisk ${IMAGE}.img
\end{lstlisting}
\FloatBarrier
\vspace{-5mm}

The \texttt{echo} commands within the parenthesis are interpreted as
key-presses in the \texttt{fdisk} commmand mode. They (i) delete the last
partition, (ii) create a new primary partition, (iii) print new partition
table and (iv) write new partition table to the image file. Use a loopback
device to make the Raspbian image available as a block device:

\begin{lstlisting}[]
# export DEV=$(sudo losetup --show -f -P ${IMAGE}.img); echo $DEV
/dev/loop0
\end{lstlisting}
\FloatBarrier
\vspace{-5mm}

\texttt{lsblk} may be used to list block devices. It appears that the image is
available as \texttt{/dev/loop0} which has two partitions \texttt{loop0p1} and
\texttt{loop0p2}:
%Use \texttt{lsblk} to view the partitions:

\begin{lstlisting}[]
# lsblk
NAME      MAJ:MIN RM  SIZE RO TYPE MOUNTPOINT
...
loop0       7:0    0  2.3G  0 loop
|-loop0p1 259:2    0   63M  0 loop
|-loop0p2 259:3    0  2.2G  0 loop
...
\end{lstlisting}
\FloatBarrier
\vspace{-5mm}

Check and resize the root partition:
\begin{lstlisting}[]
# e2fsck -f ${DEV}p2
# resize2fs ${DEV}p2
\end{lstlisting}
\FloatBarrier


%We see that the boot and root partitions starts at sector 8192 and 137216 respectivly.
%We also notice that each sector is 512 bytes.
%On newer systems, we can mount the device easier using "losetup --show -f -P IMAGE"
%Mount root partition:

\subsection{Mount image}

Browsing and altering the files in the image is easy. Simply mount the partitions.
%To browser and alter the files in the image, it is possible to mount the partitions.

Lets first mount the root partition. This is done by creating an empty directory
that we will be used as a mountpoint. We name it \texttt{root} and create it in the
working directory before mounting the root filesystem onto the mountpoint.
%After this, mount the partitions:
%After this,
%we mount the partition starting at sector 137216. To provide this offset information
%to mount, we need to find the offset in bytes. From the fdisk command we could also
%see that each sector is 512 bytes. Thus, we multiply the sector size and the sector
%start to get the offset in bytes.

\begin{lstlisting}[]
$ export ROOTDIR="${WORKDIR}/root"
$ mkdir -p ${ROOTDIR}
# mount ${DEV}p2 ${ROOTDIR}
\end{lstlisting}
\FloatBarrier
\vspace{-5mm}
%# mount -o loop,offset=$((137216*512)) ${IMAGE}.img ${ROOTDIR}

The root filesystem already has a boot directory that can be used as
the mountpoint for the boot partition.
%After this, we can also mount the boot partition inside the newly mounted
%root filesystem.
This is convenient because that partition will be mounted
on this directory when a \ac{Raspi} starts up with a memory card containing
the image. %It is therefore the natural place to mount it for later purposes.
Now, mount boot partition:
\begin{lstlisting}[]
# mount ${DEV}p1 ${ROOTDIR}/boot
\end{lstlisting}
\FloatBarrier
\vspace{-5mm}
%# mount -o loop,offset=$((8192*512)) ${IMAGE}.img ${ROOTDIR}/boot

%We can now change all the files in the disk image as desired.
It is now possible to change all files within the Raspbian image as desired.
We will take advantage of this to edit configuration files, append new files,
and even update and install packages.