\label{sec:metrics}

Once we have defined the coding schemes behavior in terms of encoding
and decoding, we proceed to describe the metrics considered in our study.
We review the energy consumption of the Raspberry Pi since this platform
is deployed at a large scale in scenarios where (i) energy is
constrained to the lifetime of the device battery and (ii) the devices
could be established in locations that are unavailable for regular
maintenance. Typical use cases of this type of scenarios are sensor
applications where devices are positioned for measurement retrieval
without any supervision for large periods fo time. In this type of
settings, both overall and per-bit energy consumption of the devices
during the encoding and decoding process are relevant parameters that
impact in the network performance for a given coding scheme. Also,
Raspberry Pi processors are based in the \ac{ARM} architecture which
is the same as in mobile devices like smartphones or tables.
Therefore, the Raspberry Pi could be used to get an estimate of
computing capabilities of a mobile device that is much more easy-deployable
and at a lower overall cost.

To complement our study, we consider the goodput defined as the ratio
of the useful delivered information at the application layer and the total
delivery time. We focus on the goodput considering only the coding process,
i.e. we assume that the application data has been properly
generated before encoding and also correctly post-processed after decoding.
The goodput is a measure for the effective processing speed since it
excludes protocol overhead but considers process

\subsection{Energy Costs}
\subsubsection{Average Consumption}
\subsubsection{Per-Bit Consumption}

\subsection{Goodput}
\subsubsection{Encoding}
\subsubsection{Decoding}

