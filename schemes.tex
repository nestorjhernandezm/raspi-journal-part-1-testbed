%!TEX root = raspi_journal.tex
\label{sec:schemes}

In this section, we present a description of the considered coding schemes
measured with our the Raspberry Pi testbed. We introduce a definition
for the primitive coding operations, e.g. encoding, decoding and recoding
(where applies) for each coding scheme. Later, we address particular schemes
which are obtained by modifying the basic coding operations that provide
better processing speeds, particularly relevant for the Raspberry Pi.

\subsection{Random Linear Network Coding}

\ac{RLNC} is an example of intra-session \ac{NC}. Here, data symbols from a
single flow are combined with each other. In this type of network coding,
all the original data packets $P_j, j \in [1,g]$, each of $B$ bits, are used
to create coded packets. In the following subsections, we describe the
basic functionalities of \ac{RLNC}.

\subsubsection{Encoding}
In \ac{RLNC}, any coded packet is a linear combination from all
the original packets. For the coding scheme, packets are seen as
algebraic entities formed as a sequence of elements from $GF(q)$,
which is a \ac{GF} of size $q$. Later, each original packet is
multiplied by a coding coefficient from $GF(q)$. The coding coefficients
are picked uniformly at random from the \ac{GF} by the encoder. To
perform the multiplication of a packet by a coding coefficient, the
coefficient is multiplied for each of the elements in the
concatenation that composes an original packet, preserving the
concatenation. Later, all resulting packets are added within the
\ac{GF} arithmetics together to generate a coded packet. Thus, a
coded packet can be written as:
%
\begin{align} \label{eq:coded_packet}
C_i  &= \bigoplus_{j=1}^{g} v_{ij} \otimes P_j ,\ \forall i \in [1,\ldots)
\end{align}

In \eqref{eq:coded_packet}, $C_i$ is the generic coded packet. In principle,
the encoder may produce any number of coded packets, but a finite
number is produced in practice given that a decoder needs only
$g$ \ac{l.i.} coded packets to decode the batch. Also, in
\eqref{eq:coded_packet}, $v_{ij}$ is the coding coefficient used in the
$i$-th coded packet and assigned to multiply the $j$-th original packet.

For indicating a receiver which packets were utilized
to create a coded one, a simple yet versatile choice is to append
its coding coefficients as a header of the coded packet. Hence, an
amount of overhead is included in every coded packet given that we
need to provide some signalling needed for decoding. The coding
coefficients overhead amount for packet $i$, $|v_i|$, can be quantified
as:
%
\begin{align} \label{eq:coded_packet_coef}
|v_i| &= \sum_{j=1}^{g} |v_{ij}| = g \times \lceil \log_{2} (q)
\rceil ~ [\mathrm{bits}]
\end{align}

\subsubsection{Decoding}
\label{sssec:decoding}
To be able to decode a batch of $g$ packets, a \ac{l.i.} set of $g$
coded packets, $C_i,\ i \in [1,g]$ is required at a decoder.
Once this set has been collected for a decoder,
the original packets can be found by computing the solution of a system
of linear equations using \ac{GF} aritmethics. Thus, we define
$\textbf{C} = \left[C_1 \ldots C_g \right]^T$,
$\textbf{P} = \left[P_1 \ldots P_g \right]^T$ and the coding matrix
$\textbf{V}$ that collects the coding coefficients for each of the $g$
coded packets, as follows:
%
\begin{align} \label{eq:coding_matrix}
\textbf{V} =
\left[
\begin{array}{c}
        v_1    \\ \hline
        \vdots \\ \hline
        v_g    \\
\end{array}
\right]
=
\left[
\begin{array}{ccc}
        v_{11} & \ldots & v_{1g} \\
        \vdots  & \ddots & \vdots  \\
        v_{g1} & \ldots & v_{gg} \\
\end{array}
\right]
\end{align}
%
Algebraically, decoding simplifies to finding the inverse of
$\textbf{V}$ in the linear system
$\textbf{C} = \textbf{V} \textbf{P}$, which can achieved using
efficient Gaussian elimination techniques \cite{fragouli2006network}.
On real applications, decoding is performed on-the-fly, e.g. the pivots are
computed as packets are progressively received, in order to minimize the
computation delay for each step.

A decoder starts to calculate and substract contributions from each
of the pivot elements, e.g. leftmost elements in the main diagonal of
\eqref{eq:coding_matrix}, from top to bottom. The purpose is to obtain
the equivalence  $\textbf{V}$ in its reduced echelon form. The steps
for reducing the matrix by elementary row operations are carried each time
a \ac{l.i.} packet is received. Once in reduced echelon form, packets
can be retrieved by doing a substitution starting from the latest
coded packet. In this way, the amount of elementary operations at the end
of the decoding process is diminished.

\subsubsection{Recoding}
As an inherent property of \ac{RLNC}, an intermediate node in the network
is able to create new coded packets without needing to decode previously
received ones from an encoding source. Therefore, \ac{RLNC} is an end-to-end
coding scheme that permits to \textit{recode} former coded packets at any
point in the network without inquiring in local decoding processing. In
principle, a recoded packet should be indistinguishable from a coded one.
Thus, we define a recoded packet as $R_i$ and consider the coding
coefficients $w_{i1}, \ldots, w_{ig}$ as the used ones to
create $R_i$. Later, a recoded packet can be written as:
%
\begin{align}
\label{eq:recoded_packet}
R_i = \bigoplus_{j=1}^{g} w_{ij} \otimes C_j  ,\ \forall i \in [1,\ldots)
\end{align}

In \eqref{eq:recoded_packet}, $w_{ij}$ is the coding coefficient used
in the $i$-th recoded packet and assigned to multiply a previously coded
packet $C_j$. These coding coefficients are again uniformly and randomly chosen
from $GF(q)$. Still, each of the $w_{ij}$ are not the appended recoding
coefficients for signalling since the they will only retrieve previously
coded packets. The analytical expression for the appended recoding
coefficients is obtained by calculating recoded packets as function of the
original packets. Therefore, we define the local coding matrix $\textbf{W}$
in the same way as it was made for $\textbf{V}$. Thus, the local coding matrix
can be written as:
%
\begin{align} \label{eq:local_coding_matrix}
\textbf{W} =
\left[
\begin{array}{c}
        w_1    \\ \hline
        \vdots \\ \hline
        w_g    \\
\end{array}
\right]
=
\left[
\begin{array}{ccc}
        w_{11} & \ldots & w_{1g} \\
        \vdots  & \ddots & \vdots  \\
        w_{g1} & \ldots & w_{gg} \\
\end{array}
\right]
\end{align}
%
With the definitions from \eqref{eq:coding_matrix} and
\eqref{eq:local_coding_matrix}, the recoded packets
$\textbf{R} = \left[R_1 \ldots R_g \right]^T$ are written as
$\textbf{R} = (\textbf{W} \textbf{V}) \textbf{P}$. Here, we recognize
the relationship between original and recoded packets. The resulting
coding matrix is simply the multiplication of matrices $\textbf{W}$ and
$\textbf{V}$. Denoting ${(\textbf{W} \textbf{V})}_{ij}$ as the element
in the $i$-th row and $j$-th column of $(\textbf{W} \textbf{V})$, this
term is the resulting coding coefficient used to create $R_i$ after
encoding the original packet $P_j$ recoding it locally in an intermediate
node. In this case, we have assumed that recoding has taken place at the
closesBy doing some algebra, these terms can be verified to be computed
as:
%
\begin{align}
\label{eq:appended_coded_coefficients}
{(\textbf{W} \textbf{V})}_{ij} = \sum_{k=1}^{g} w_{ik} v_{kj}\ ,\ \forall i,j \in [1,g] \times [1,g]
\end{align}
%
Similarly as with encoding, any decoder that collects a set
of $g$ \ac{l.i.} recoded packets with their respective $w_i$, will be able to
decode the data as mentioned before in Section \ref{sssec:decoding}. Finally,
the appended coding coefficients for decoding

\subsection{Sparse Random Linear Network Coding}

In \ac{SRLNC}, instead of considering all the packets to create a coded
packet like \ac{RLNC} does, here the encoder sets more coding coefficients to
zero when creating a new coded packet. In practice this can be implemented in
a couple of ways.

%One way to accomplish this is to add a Bernoulli random
%variable, $\text{B}(d)$, as in \ref{eq:sparse_coded_packet}:
The simplest way to accomplish this is to add a Bernoulli random
variable, $\text{B}(d)$, as in \ref{eq:sparse_coded_packet}:

\begin{align} \label{eq:sparse_coded_packet}
    %C_i  &= \bigoplus_{j=1}^{g} \text{B}(d)_{j} \, v_{ij} \otimes P_j ,\ \forall i \in [1,g] ,\
    C_i  &= \bigoplus_{j=1}^{g} \text{B}(d) \, v_{ij} \otimes P_j ,\ \forall i \in [1,g] ,\
    v_{ij} > 0,\
    d \in
    \begin{cases}
        {(0,0.5]} & \text{for } q = 2 \\
        {(0,1]} & \text{for } q > 2
    \end{cases}
\end{align}

We call $d$ the coding density.
Reducing $d$ enables the encoder to decrease the average number of packets that
are mixed into each coded packet. Thereby reducing the complexity of the
encoder that needs to mix less packets, but also on the decoder that in
contrast needs to eliminate less nonzero coding coefficients. The catch of this
approach is that coded packets produced by the encoder becomes more \ac{l.d.}
on each other as the coding complexity is reduced. This incurs in additional
channel overhead.

Generating a coded packet does not imply that it contains any data at all. I.e.
$ \left[ B(d)_{1}, \ldots, B(d)_{g} \right] = \mathbf{0}$.
In that case, one would have to trash the coded packet and generate a new one.

Most coded packets that are produced by the encoder are typically very sparse.
We will therefore rely on a slightly different implementation in KODO
\ref{eq:sparse_coded_packet2} that proved faster for low densities. It is best
expressed as:

\begin{align} \label{eq:sparse_coded_packet2}
    C_i  &= \bigoplus_{j=1}^{\floor{\text{Binom}(g,d)}} v_{im} \otimes P_{m} ,\
    %L \subseteq \{1,\dots,g\}: |L|=k ,\
    \forall i \in [1,g] ,\
    v_{ij} > 0
\end{align}
where
$m = \text{choice}(\{1,\dots,g\}, \text{size}=k, \text{replace}=\text{False})$.
\fxnote{Can this be represented smarter?}


%To solve this, one could trash the coded packet and generate a new coded
%packet. This is not the most efficient because KODO needs to count the
%the number of nonzero coding coefficients in each coded packets to eliminate
%all zero packets.
%
%This is the most correct approach, but the most efficient approach although it is mathematically .

It is often desired to express the coding density as $d=\frac{k}{g}$. We call
this notation $k-sparse$ where $k$ is the average number of packets combined to
produce a coded packet.

%For low coding density, the encoder might generate a coded packet that consists
%of no coded packets. I.e. the packet consists of all zeroes






%In \ac{SRLNC}, instead of considering all the packets to create a coded
%packet like \ac{RLNC} does, here the encoder sets some coding coefficients to
%zero when creating a new coded packet. This permits to reduce the complexity
%of both the encoding and decoding operations since the encoder (or decoder)
%requires less operations to process the data. Therefore, a new code parameter
%to control the sparsity, the code density $d$, is included to vary the
%number of non-zero packet that are combined. Here are various approaches to
%set the density.
%
%A first method is to specify the number of zeros in a given
%generation size and later generate coded packets with this amount. For example:
%16 zeros in a coding vector of 64 packets where the position of zeros are
%modified at random for every coded packet generated. Another method is to
%define an average number of zeros in the coding vector. Therefore, more or
%less zeros are exactly generated for each coded packet in the process.
%However, if not carefully made, using a sparse coding scheme might increase
%the probability of generating \ac{l.d.} packets which incurs both overhead
%from redundant information and processing delay.

\subsection{Tunable Sparse Network Coding}

\ac{TSNC} can be considered an extension to \ac{SRLNC}. The overall idea is
not only for the encoder to generate sparse coded packets, but also to
choose the sparsity of packets wiser. That is, as the decoder accomulates more
coded packets it becomes more and more likely that the next received
coded packet will be \ac{l.d.}.

The idea of \ac{TSNC} is therefore to gradually increase the coding density
as the decoderes \ac{DOF} increases. This enables TSNC to save a lot of
complexity particulary in the beginning of transmissions, but also to
tailor the coding delay. I.e. approximate the number of coded packets a
decoder needs to receive to decode a generation of $g$ packets. We call
this number the budget, $B$.

Provided a desired budget, $B \geq g$, and the decoders \ac{DOF} it can be
estimated how sparse the coded packets should be to finish a transmission in
time.

In our implementation, it is the encoders responsibility to estimate the coding
density. We use feedback to provide the encoder a better estimate of the
decoders \ac{DOF} at pre-defined points. Namely, when the decoder obtains the
following \ac{DOF}'s:

\begin{equation}\label{eq:feedback_dofs}
    s(k) = g\cdot \frac{2^k -1}{2^k},  \; \text{for } k=0,1,2,\dots
\end{equation}
where $k$ is the $k$-th feedback packet that is transmitted during a generation.

The encoder considers the time between feedback packets as regions. It then
estimates a fixed coding density for each region when a feedback packet is
received.

Before the coding density can be estimated it is essential that the encoder
has a good estimate of the decoders \ac{DOF}, but also the remainder of the
total budget. The feedback scheme \ref{eq:feedback_dofs} roughly splits the
remaining \ac{DOF} in halves. This is also the case for the budget that is
split equally among the two new regions. The very last region will be assigned
the full remainder of the total budget.

\subsection{Network Coding Implementation for Multicore Architectures}
\label{sub:implementation-multicore}

The operations needed to encode and decode the data are, in general, similar. To
encode packets, the encoder needs to perform the matrix multiplication
$\textbf{C} = \textbf{V} \cdot \textbf{P}$. On the other hand, decoding the
information requires the decoder to find $\textbf{V}^{-1}$ and to perform the
matrix multiplication $\textbf{P} = \textbf{V}^{-1} \cdot \textbf{C}$. In both
cases, a matrix multiplication is needed. Therefore, to make a practical
implementation of network coding, it is valuable to find a way to optimize the
matrix multiplications operations for multicore architectures.

When designing multithreaded algorithms for network coding operations, it is
possible to implement the decoding by combining the matrix inversion and the
matrix multiplication, e.g., performing the Gauss-Jordan algorithm over the
coding matrix $\textbf{V}$ while performing, in parallel, row operations on the
coded data $\textbf{C}$. However, the parallelization of such operations provide
speeds up that are limited for small block sizes ($\leq 2048$ bytes). The reason
is that to operate in a parallel fashion over the same coded packet $C_i$
requires strained synchronization. In~\cite{5061951} and~\cite{4262451} the
parallelization of the row operations are optimized for \ac{GPU} and \ac{SMP}
systems respectively.

Therefore, to overcome the constraints of a tight synchronization, it is
preferable to explicitly invert the matrix $\textbf{V}$, and then take advantage
of optimizations for matrix multiplications. The authors implemented
in~\cite{wunderlich2015network} an algorithm that adopts the ideas of efficient
\ac{BLAS} operations~\cite{lawson1979basic} reimplementing them for finite
fields operations. The implemented algorithm tries to be cache efficient
maximizing the number of operations performed over a block of fetched data. This
is done by operating on square sub-blocks of the matrices, which exploits the
spatial locality of the data, at least for $\mathcal{O}(n^3)$
algorithms~\cite{golub2012matrix}. The optimal size of the block is
architecture-dependent. The ideal block has a number of operands that fit into
L1 cache, and it is a multiple of the \ac{SIMD} operations size.

The main idea of the implemented algorithm is to represent each one of the sub-
block matrix operations as a base, \textit{kernel}, operation that can be
optimized, e.g., using \ac{SIMD} operations. Each of these kernels, at the same
time, can be represented as a task with inputs and outputs in memory. If the
algorithm is formulated conventionally, and the data operations on the sub-
blocks are reordered, then it is possible to resolve data dependencies with a
\ac{DAG}. For the algorithm, a dependency means, for example, that a sub-block
of the matrix in memory cannot be read unless a previous writing operation was
performed there. When the \ac{DAG} is determined, then, a scheduler assigns to
the available cores the individual kernel operations that have their
dependencies fulfilled. The benefit of this algorithm is that the
synchronization relies only on data dependencies, and it does not requires the
insertion of artificial synchronization points. Using this technique, the matrix
inversion is performed using an algorithm based on LU
factorization~\cite{Dongarra:2011:HPM:2132876.2132885}, and the matrix
multiplication is performed by making the various matrix-matrix multiplications
of the sub-blocks.
