\label{sec:testbed_nfs}


\subsection{NFS}


\begin{lstlisting}[]
# export NFSBOOTIMAGE="${WORKDIR}/nfsboot"
# export NFSROOTIMAGE="${WORKDIR}/nfsroot"
\end{lstlisting}
\FloatBarrier
\vspace{-5mm}

Extract root partition from image
\begin{lstlisting}[]
# export DEV=$(sudo losetup --show -f -P ${IMAGE}.img); echo $DEV
# dd if=${DEV}p1 of=${NFSBOOTIMAGE}.img bs=4M
# dd if=${DEV}p2 of=${NFSROOTIMAGE}.img bs=4M
\end{lstlisting}
\FloatBarrier
\vspace{-5mm}


Mount nfs root, boot, and home directories
\begin{lstlisting}[]
$ export NFSROOTDIR="${WORKDIR}/nfsroot"
$ mkdir -p ${NFSROOTDIR}
# mount ${NFSROOTIMAGE}.img ${NFSROOTDIR}
# mount ${NFSBOOTIMAGE}.img ${NFSROOTDIR}/boot
\end{lstlisting}
\FloatBarrier
\vspace{-5mm}


Add the below to NFSROOTDIR/boot/cmdline.txt
\begin{lstlisting}[]
ip=dhcp root=/dev/nfs nfs-options=hard,intr,ro
\end{lstlisting}
\FloatBarrier
\vspace{-5mm}


\Suppressnumber\begin{lstlisting}[]
<@\textcolor{gray}{\$\{NFSROOTDIR\}/etc/initramfs-tools/hooks/hooks-rootimg}@>
<@\textcolor{gray}{
---------------------------------------------------------------}
\Reactivatenumber @>
#!/bin/sh

. /usr/share/initramfs-tools/scripts/functions
. /usr/share/initramfs-tools/hook-functions

copy_exec /sbin/losetup
\end{lstlisting}
\FloatBarrier
\vspace{-5mm}


\Suppressnumber\begin{lstlisting}[]
<@\textcolor{gray}{\$\{NFSROOTDIR\}/etc/initramfs-tools/hooks/init-premount-rootimg}@>
<@\textcolor{gray}{
---------------------------------------------------------------}
\Reactivatenumber @>
#!/bin/sh

PREREQ=""
prereqs()
{
   echo "$PREREQ"
}

case $1 in
prereqs)
   prereqs
   exit 0
   ;;
esac

. /scripts/functions
log_begin_msg "Starting rootimg"
log_end_msg

# Setup loop device
losetup /dev/loop0 /imgdir/root.img
\end{lstlisting}
\FloatBarrier
\vspace{-5mm}














IS SETTING UP THE SD-CARD REALLY THE SAME PROCEDURE AS FOR A LOCAL FILESYSTEM?

% Define variable
% export IMAGE="2016-05-27-raspbian-jessie-lite"
% export WORKDIR="${HOME}/Raspbian"
\begin{lstlisting}[]
$ export NFSDEV="/dev/mmcblk0"
\end{lstlisting}
\FloatBarrier
\vspace{-5mm}

\begin{lstlisting}[]
<@\Suppressnumber @># (echo o;                                      # Create DOS partition table
    echo n; echo p; echo 1; echo ; echo +64M;   # Create boot partition
    echo t; echo b;                             # Set boot partition type to FAT
    echo n; echo p; echo 2; echo ; echo ;       # Create home partition
    echo w                                      # Write table to disk
    ) | sudo fdisk ${NFSDEV} <@\Reactivatenumber @>
\end{lstlisting}
\FloatBarrier
\vspace{-5mm}

Print partition table (SHOULD THE BOOT FLAG HAVE BEEN SET TO PARTITION P1?)
\begin{lstlisting}[]
# fdisk -u sectors -l ${NFSDEV}
Disk /dev/mmcblk0: 3.7 GiB, 3965190144 bytes, 7744512 sectors
Units: sectors of 1 * 512 = 512 bytes
Sector size (logical/physical): 512 bytes / 512 bytes
I/O size (minimum/optimal): 512 bytes / 512 bytes
Disklabel type: dos
Disk identifier: 0x00e61b04

Device         Boot  Start     End Sectors  Size Id Type
/dev/mmcblk0p1        2048  133119  131072   64M  b W95 FAT32
/dev/mmcblk0p2      133120 7744511 7611392  3.6G 83 Linux
\end{lstlisting}
\FloatBarrier
\vspace{-5mm}

Format partitions
\begin{lstlisting}[]
# mkfs.vfat -n BOOT ${NFSDEV}p1
# mkfs.ext4 -L HOME ${NFSDEV}p2
\end{lstlisting}
\FloatBarrier
\vspace{-5mm}



%Mount image:
%\begin{lstlisting}[]
%# export DEV=$(sudo losetup --show -f -P ${IMAGE}.img); echo $DEV
%# mount ${DEV}p2 ${ROOTDIR}
%# mount ${DEV}p1 ${ROOTDIR}/boot
%\end{lstlisting}
%\FloatBarrier
%\vspace{-5mm}





Copy boot directory:
\begin{lstlisting}[]
# cp -r ${ROOTDIR}/boot/* ${NFSBOOTDIR}/ 
\end{lstlisting}
\FloatBarrier
\vspace{-5mm}



Prepare home directory (Maybe it is better just to copy the content in home from the
image's root partition):
\begin{lstlisting}[]
# install -d -o 1000 -g 1000 -m 700 ${HOMEDIR}/pi && sync
\end{lstlisting}
\FloatBarrier
\vspace{-5mm}



\begin{lstlisting}[]
# mount ${NFSDEV}p1 ${NFSROOTDIR}/boot
# mount ${NFSDEV}p2 ${NFSROOTDIR}/home
\end{lstlisting}
\FloatBarrier
\vspace{-5mm}


Unmount directories:
\begin{lstlisting}[]
# umount ${NFSBOOTDIR} ${NFSHOMEDIR}
\end{lstlisting}
\FloatBarrier
\vspace{-5mm}

