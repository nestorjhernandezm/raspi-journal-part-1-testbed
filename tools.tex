
%\subsection{Kodo cross-compilation: From your PC to the Raspberry Pi}
\label{sec:tools}
Within the testbed everyday use, there exists frequent tasks that require
a set of various commands in a given \ac{Raspi}. This could be tedious, prone
to errors and time consuming to realize everytime the task is required to be
made. Therefore, in this section we introduce a set of tools that help to
automate and monitor routinary tasks execution in the \ac{Raspi}s and show
relevant example commands with them. To be able to run all the following
commands, it is required to have \ac{SSH} connectivity with the \ac{Raspi}s,
otherwise the commands need to be run locally on a \ac{Raspi} making
necessary to use a keyboard and a monitor. The testbed
administrator needs to put the memory cards in the \ac{Raspi}s and turn them
on, for them to be able to boot. The devices should now be bootable.


\subsection{Fabric}
Controlling multiple devices using \ac{SSH} from a single \ac{PC}
often leads to many repetitive tasks. Among these, we can mention:
(i) rebooting a set of devices, (ii) installing applications in multiple
devices and (iii) copying files to / from multiple devices.
Fabric~\cite{python_fabric} provides a Python library that simplifies the
management of working with many devices from a single \ac{PC}. First, the
testbed administrator creates a directory to hold the \texttt{Fabric} source
code:

\begin{lstlisting}[]
$ export CODEDIR="${HOME}/code"
$ mkdir -p ${CODEDIR}
$ cd ${CODEDIR}
\end{lstlisting}
\FloatBarrier
\vspace{-5mm}

Then, the \texttt{\$\{CODEDIR\}/fabfile.py} file below provides a script with
some basic functionalities that can perform the few items above (i-iii).
In general, other administrators may require different functionalities
but that goes out of the scope of this work. The following
file serves as a starting boilerplate:

% Exit chroot, umount, and write to memory card
\Suppressnumber\begin{lstlisting}[]
<@\textcolor{gray}{\$\{CODEDIR\}/fabfile.py}@>
<@\textcolor{gray}{
-----------------------------------------}
\Reactivatenumber @>
from fabric.api import env, task, sudo
# Python Fabric script to run commands on multiple hosts through ssh
#
# Run script as 'fab <task>', where <task> is one of the scripts functions
# marked as a tesk. The task marked as 'default' will be run if <task> is not
# specified

env.hosts = ['10.0.0.100','10.0.0.101','10.0.0.102']
env.user = 'pi'
env.password = 'raspberry'

@task
def reboot():
    """ Reboot device """
    sudo('reboot', quiet=True)

@task
def install(program):
    """
    Install a program
    program: program name
    """
    result = sudo('apt-get install -y {}'.format(program), quiet=True)
    print(result)

@task
def push(src,dst):
    """
    Copy file to device
    src: source file path
    dst: destination file path
    """
    put(src, dst)

\end{lstlisting}
\FloatBarrier
\vspace{-5mm}

The previous \texttt{fabfile} shows three functions that perform our
example tasks. These functions utilize variables and subsequent functions
from the Fabric \ac{API} such as \texttt{env}, \texttt{task} and \texttt{sudo}
among others. Each of these \ac{API} functions permits to define environment
variables, create the administrator tasks through decorators or run the
mentioned task in \texttt{sudo} mode, respectively.
When a task is called from the terminal, Fabric searches the directory for
the \texttt{fabfile.py} file and executes the desired task. The syntax for
executing a task with arguments is in the form
\texttt{fab <TASK>:arg1,arg2,...}. In what follows, we denote the
\ac{IP} address of a generic \ac{Raspi} for test as \texttt{<RASP\_IP>}.
The executions from the terminal of some of these commands are shown as
follows:

\begin{lstlisting}[]
$ fab reboot
[<RASP_IP>] Executing task 'reboot'

Done.
Disconnecting from <RASP_IP>... done.
$ fab install:tmux
[<RASP_IP>] Executing task 'install'
...
The following NEW packages will be installed:
  tmux
0 upgraded, 1 newly installed, 0 to remove and 0 not upgraded.
...
Preparing to unpack .../archives/tmux_1.9-6_armhf.deb ...
Unpacking tmux (1.9-6) ...
Processing triggers for man-db (2.7.0.2-5) ...
Setting up tmux (1.9-6) ...

Done.
Disconnecting from <RASP_IP>... done.
\end{lstlisting}
\FloatBarrier
\vspace{-5mm}

The first function above reboots the \ac{Raspi}s in the lists of hosts and
the second function installs a program given by an argument. For the
connection to the devices, Fabric calls the Paramiko~\cite{python_paramiko} module from Python
to make a \ac{SSH} connection. For this to work properly, the Paramiko
version needs to be higher than or equal to 1.15.1. If not available,
the \ac{SSH} connections from Fabric may fail. In case of any problems,
some instructions for updating the Paramiko package are available in
Section~\ref{sec:paramiko} of the Appendices. This is a standard recommendation
the Fabric troubleshooting guide~\cite{2016fabricsupport}.

After a successful \ac{SSH}
connection is made, in the previous two commands towards the \ac{Raspi},
Fabric employs the \ac{Raspi}'s \texttt{reboot} and \texttt{apt-get}
commands in \texttt{sudo} mode to do the required tasks. Below it is shown
an example for the \texttt{push} task which uses two arguments. Here we
copy \texttt{my\_file} from the testbed administrator \ac{PC} to a test
host \ac{Raspi}:

\begin{lstlisting}[]
$ fab push:"${CODEDIR}/my_file",'~/'
[<RASP_IP>] Executing task 'push'
[<RASP_IP>] put: /home/<USER>/code/my_file -> /home/pi/my_file

Done.
Disconnecting from <RASP_IP>... done.
\end{lstlisting}
\FloatBarrier
\vspace{-5mm}

To control a large set of devices, we simply need to include them in the
\texttt{env.hosts} list in the \texttt{\$\{CODEDIR\}/fabfile.py} file.
Fabric has many other functionalities that are useful in controlling a
large set of \ac{Raspi}s. For example, we may extract files or run
automated experiments. The included functionalities in the
\texttt{\$\{CODEDIR\}/fabfile.py} file will depend on the requirements of the
testbed administrator.

%execute a script function by calling "fab install:feh"
\subsection{Long running jobs using SSH}
There are times when a task may need to run for several hours or even days
in the \ac{Raspi}s, particularly when related to simulations or measurement
campaigns. For this purpose, it might be necessary to keep open a \ac{SSH}
connection in the \ac{Raspi}s without risking that the connection will
be interrupted and a given \ac{Raspi} terminating the task.

There are methods to enable the \ac{Raspi}s to continue running
applications although the connection is terminated either on purpose or
unexpectedly. One method is to run programs within a \texttt{screen} session.
\texttt{screen} enables a user to run applications within a shell
window, a \texttt{screen} session, which does not terminate even with
connectivity interruptions. Users can attach and detach from a
\texttt{screen} session
as desired. The following procedure presents how to use \texttt{screen} with
\ac{SSH} to: (i) Login to a generic \ac{Raspi}, (ii) open a screen session,
(iii) execute an example command, (iv) detach from the screen session, (v)
terminate the \ac{SSH} connection, (vi) login to the \ac{Raspi} again
and (vii) attach to screen session to see the program still running.
From the testbed administrator \ac{PC}, we start by establishing a \ac{SSH}
connection to a \ac{Raspi} and open a screen session:

\begin{lstlisting}[]
$ ssh pi@<RASP_IP>
...
pi@<RASP_IP>'s password:

The programs included with the Debian GNU/Linux system are free software;
the exact distribution terms for each program are described in the
individual files in /usr/share/doc/*/copyright.

Debian GNU/Linux comes with ABSOLUTELY NO WARRANTY, to the extent
permitted by applicable law.
Last login: Tue Jul 12 13:04:31 2016

$ screen
Screen version 4.02.01 (GNU) 28-Apr-14
...
[Press space or Return to end.]
\end{lstlisting}
\FloatBarrier
\vspace{-5mm}

To enter in the \texttt{screen} session after the introduction message, we
have to press either the \texttt{Space} or \texttt{Return} key in the
keyboard to clear the shell. After doing so, we should be in a
\texttt{screen} session although its appearance is the same as a regular
terminal shell. Inside this example session, we execute a program that
never ends:

\begin{lstlisting}[]
$ top
\end{lstlisting}
\FloatBarrier
\vspace{-5mm}

The \texttt{top} command simply shows continuously the table of processes
executed on the \ac{Raspi} like in any Linux distribution. When \texttt{top}
is running, we first press \texttt{Ctrl+a} and later \texttt{Crtl+d}
in the keyboard to detach from the \texttt{screen} session. We now
terminate the \ac{SSH} connection and login again to verify that the
\texttt{top} command is still running. Without using \texttt{screen},
the \texttt{top} program should terminate since its hosting shell was
terminated. To log out, we run:

\begin{lstlisting}[]
$ exit
logout
Connection to <RASP_IP> closed.
$ ssh pi@<RASP_IP>
\end{lstlisting}
\FloatBarrier
\vspace{-5mm}

Now that we are logged in to the \ac{Raspi} again, we first check the
available detached sessions by running:
\begin{lstlisting}[]
$ screen -list
There is a screen on:
  824.pts-0.raspXX (07/12/16 13:17:30) (Detached)
1 Socket in /var/run/screen/S-pi.
\end{lstlisting}
\FloatBarrier
\vspace{-5mm}

From the command output, we can see that the session is still running in our
generic \ac{Raspi} number \texttt{XX} and that no user is currently attached
to the session. To attach to the session, we execute:

\begin{lstlisting}[]
$ screen -r 824.pts-0.raspXX
\end{lstlisting}
\FloatBarrier
\vspace{-5mm}

After attaching again, we should see \texttt{top} still running.
\texttt{screen} has more functionalities that can be used in this
or other contexts but that is outside the scope of this work.
To terminate the \texttt{screen} session, first terminate \texttt{top}
by pressing \texttt{q} in the keyboard.
Once \texttt{top} is terminated,
%Once out of the \texttt{top} output,
we need to type \texttt{exit} two times in order to first exit the
\texttt{screen} session and then terminate the \ac{SSH} connection.
An output should be as follows:


\begin{lstlisting}[]
[screen is terminating]
pi@<RASP_IP>:~ $ exit
logout
Connection to <RASP_IP> closed.
$
\end{lstlisting}
\FloatBarrier
\vspace{-5mm}
